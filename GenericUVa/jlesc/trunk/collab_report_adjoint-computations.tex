

\subsubsection{Checkpointing strategies for adjoint computations \ongoing}
\label{sec.report.adjoint} 

% start new section for referencing, please use the name of your bibfile here instead of report2015 (.bib)
\begin{refsection}[adjoint-computations]

\paragraph{Participants}~\\

\begin{itemize}
	\item Guillaume Aupy (INRIA)
	\item Julien Herrmann (INRIA)
	\item Paul Hovland (ANL)
	\item Yves Robert (INRIA)
\end{itemize}


\paragraph{Research topic and goals}~\\
%List research topic and goals.
The need to efficiently compute the derivatives of a function arises frequently
in many areas of scientific computing, including mathematical optimization,
uncertainty quantification, and nonlinear systems of equations. When the first
derivatives of a scalar-valued function are desired, so-called adjoint
computations can compute the gradient at a cost equal to a small constant times
the cost of the function itself, without regard to the number of independent
variables.
In this work we want to optimize the computational time of such functions when
the memory is not sufficient to hold all  necessary data~\cite{Heimbach20051356}.

Previously, optimal algorithms for adjoint computations were known only for a
single level of checkpoints with no writing and reading costs; a well-known
example is the binomial checkpointing algorithm of Griewank and
Walther~\cite{griewank1992achieving,griewank2000algorithm}. Stumm and Walther
extended that binomial checkpointing algorithm to the case of two levels of
checkpoints, but they did not provide any optimality
results~\cite{stumm2009multistage}.

Goals:
\begin{enumerate}
	\item Computing efficient schedules for the offline adjoint problem with two
storage levels (memory and disks);
	\item Computing efficient schedules for the online adjoint problem (when the
number of operations to perform is not known before execution);
	\item Computing efficient schedules in the presence of failures;
	\item Implementing thee algorithms into adjoint computation software.
\end{enumerate}


\paragraph{Results for current year}~\\

We have designed the first optimal algorithm for the offline adjoint problem with two
storage levels. A prototype evaluation showed that we save more than 60\% on
the execution time on realistic adjoint computations~\cite{aupy:hal-01147155}.

%We are currently finish
%Write a few paragraphs on main results for current year. Describe relevant publications here~\cite{aupy:hal-01147155}.

\paragraph{Visits and meetings}~\\
%List visits and meetings (planned and done).
%Visits done:
\emph{Past visits and meetings}
\begin{itemize}
	\item[] Guillaume Aupy visited Paul Hovland at Argonne National Laboratory
from Sept. 2014 to Dec. 2014 .
\end{itemize}
\emph{Planned visits and meetings}
\begin{itemize}
	\item[] Paul Hovland will visit Guillaume Aupy and Yves Robert from
June 22nd, 2015 to June 26th, 2015.
	\item[] Guillaume Aupy and Paul Hovland plan to attend the $17^{th}$
AutoDiff Workshop in July 2015 to present the first results.
\end{itemize}

\paragraph{Impact and publications}~\\

% print list of publications containing the ``own'' keyword (for publications done within this project and year)
\printbibliography[heading=none,keyword=own]


%\begin{itemize*}
%    \item done this
%    \item impact there
%\end{itemize*}

\paragraph{Person-Month efforts}~\\

%This is very important for the JLESC activity report. 
%Detail person-months spent by both permanent and temporary researchers 
%who worked for the collaboration.
\begin{itemize}
	\item {\bf Guillaume Aupy}, temporary researcher, 8 months
	\item {\bf Julien Herrmann}, temporary researcher, 4 months
	\item {\bf Paul Hovland}, permanent researcher, 1 month
	\item {\bf Yves Robert}, permanent researcher, 1 month
\end{itemize}


\paragraph{Future plans}~\\
%Describe future plans here.

As stated above, we aim at achieving the following goals:
\begin{enumerate}
	\item Computing efficient schedules for the online adjoint problem (when the
number of operations to perform is not known before-hand);
	\item Computing efficient schedules in the presence of failures;
	\item Implementing those algorithms into adjoint computation software.
\end{enumerate}

\paragraph{References}~\\

%Provide a few bibliographical references here.

% print list of publications not from this project but of other relevance
\printbibliography[heading=none,notkeyword=own]
%\bibfile{mybeautifulproject}
\end{refsection}
