
\subsubsection{Smart In Situ Visualization}
\label{sec.report.tp} % add label, replace xx with your initials

% start new section for referencing, please use the name of your bibfile here instead of report2015 (.bib)
\begin{refsection}[report2015_tp]

\paragraph{Participants}~\\

\begin{itemize*}
\item Lokman Rahmani, ENS Rennes, PhD student
\item Hadrien Croubois, ENS Lyon, Visiting scholar
\item Matthieu Dorier, ANL, Postdoc
\item Matthieu Dreher, ANL, Postdoc
\item Bruno Raffin, Inria, Senior Researcher
\item Tom Peterka, ANL, Senior Researcher
\item Luc Bougé, ENS Rennes, Professor
\item Gabriel Antoniu, Inria, Senior Researcher
\item Roberto Sisneros, NCSA, Senior Researcher
\item Justin Wozniak, ANL, Senior Researcher
\end{itemize*}

\paragraph{Research topic and goals}~\\

The increasing gap between computational power and I/O performance in new supercomputers drives
a shift from an offline approach of data analysis to an inline approach, termed in situ
visualization (ISV). While many parallel visualization tools now provide ISV, the trend has
been to feed such software with what previously was large dumps of raw data, and let them
render everything at the highest possible resolution. This leads to a potentially large
performance impact in simulations that support ISV, in particular when ISV is performed
interactively. We are researching smarter methods of performing ISV. Our approach aims
to detect potentially interesting regions in the generated dataset in order to feed ISV
frameworks with only a subset of the data produced by the simulation. While this method
mitigates the load on ISV frameworks, making them more efficient and more interactive, it also
helps scientists focus on the relevant part of their data.

\paragraph{Results for current year}~\\

We are developing a smart ISV framework that allows for data reduction based on its
potential scientific value, controlled by the user and transparent to the simulation and the
analysis and visualization tools. We implemented such a solution for data generated with a spatial
decomposition of the simulation domain across processes. We choose to define the scientific
value of the filtered data as the quality of visualization (QoV) of the images generated by
rendering it, and we use the structural similarity index metric (SSIM)~\cite{ssim} to objectively
quantify QoV loss. SSIM is a reference metric to quantify the human perception of loss quality
in a compressed image. It considers the structural differences of an image with respect to a
reference one.

Data reduction in our contribution is based on a generic, simple yet efficient estimation of
relevant data in terms of local, spatial variability. We are investigating different metrics
based on statistics, information theory and image processing, to compute this local
variability. The resolution at which a region of data is rendered then depends on the local
value of these metrics.  We evaluated Smart ISV through experiments on the French Grid’5000
using the CM1 atmospheric simulation~\cite{Bryan:2002}, with VisIt as a visualization backend
and Damaris/Viz as the interface between CM1 and VisIt. Damaris/Viz allows us to define our
data filters as plugins, thus making the “Smart” component of ISV transparent to both CM1 and
VisIt.  Results show an up to 40\% gain in rendering performance with limited loss in QoV, when
using our data reduction algorithms.

\paragraph{Visits and meetings}~\\

Past visits:
\begin{itemize*}
\item October - December 2014: Lokman Rahmani 3 month - visit to ANL
\item July 2014 - June 2015: Hadrien Croubois 12 month visit to ANL
\end{itemize*}

Planned visits:
\begin{itemize*}
\item June 2015: will meet with Gabriel Antoniu, Bruno Raffin, Justin Wozniak, Matthieu Dorier,
  Matthieu Dreher in Barcelona to discuss progress of Smart Visualization and plan next project
  in Decaf-Swift integration.
\item September 2015: will meet with Gabriel Antoniu at Cluster 2015 in Chicago, IL to update
  progress
\item November 2015: will meet with Gabriel Antoniu and Roberto Sisneros at SC15 in Austin, TX to plan next steps
\item September 2015 - August 2016: extended invitation to Pierre-Louis Guhur (ENS Cachan) for
  12 month visit to ANL
\end{itemize*}

\paragraph{Impact and publications}~\\

We are preparing a manuscript for an upcoming systems or visualization
conference~\cite{rahmani15}.

\printbibliography[heading=none,keyword=own]

Further impact:

\begin{itemize*}
\item An approach for Smart ISV using iteration-level data reduction based on the scientific
  value of the data;
\item A domain-independent approach for estimating the scientific value of a dataset;
\item A comparative discussion of several metrics for relevant data selection, allowing to
  implement the aforementioned approach;
\item A prototype implementation of the proposed data selection and reduction algorithms into
  the Damaris/Viz ISV framework as an external plugin.
\item An evaluation showing a positive impact of our approach to smart ISV on the performance
  with little loss in quality of visualization, using a real atmospheric simulation.
\end{itemize*}

\paragraph{Person-Month efforts}~\\

\begin{itemize*}
\item Lokman Rahmani: 6 months
\item Matthieu Dorier: 1 month
\item Tom Peterka: 1 month
\item Luc Bougé: 0.5 month
\item Gabriel Antoniu: 0.5 month
\item Roberto Sisneros: 0.5 month
\end{itemize*}

\paragraph{Future plans}~\\

As of today, experiments were run with climate simulations only. Also, scientific relevance of
data was measured by their perception by human visual system. The smart ISV capabilities was
integrated in Damaris/Viz only but can be extended to any other ISV frameworks with plugin
management system and internal data model. We are considering extending the data
reduction process to other type of simulation, not based on space-partitioned datasets. An
advanced study of alternative selection metrics can be also interesting. Finally, to take full
profit of Damaris/Viz we can provide a VisIt plugin that interact dynamically with the data
reduction plugin at simulation runtime.

\paragraph{References}~\\

% print list of publications not from this project but of other relevance
\printbibliography[heading=none,notkeyword=own]

\end{refsection}
