
\subsubsection{Energy Efficiency and Load Balancing}
\label{sec.report.jfm} % add label, replace xx with your initials

% start new section for referencing, please use the name of your bibfile here instead of report2015 (.bib)
\begin{refsection}[loadbalancing]

\paragraph{Participants}~\\

\begin{itemize}
\item Rafael Keller Tesser, PhD student UFRGS, associated INRIA team ExaSE
\item Edson Luiz Padoin, PhD student, UFRGS, associated INRIA team ExaSE
\item Philippe Navaux, Professor, UFRGS, associated INRIA team ExaSE
\item Celso Mendes, NCSA
\item Sanjay Kale, UIUC
\item Jean-François Méhaut, Professor, INRIA Grenoble, Corse INRIA team, associated INRIA team ExaSE
\end{itemize}

\paragraph{Research topic and goals}~\\

The power consumption of High Performance Computing (HPC) systems is
an increasing concern as large-scale systems grow in size and,
consequently, consume more energy.  In response to this challenge, we
propose new energy-aware load balancers that aim at reducing the
energy consumption of parallel platforms running imbalanced scientific
applications without degrading their performance. Our research
explores dynamic load balancing, low power manycore platforms and DVFS 
techniques  in order to reduce power consumption.

\paragraph{Results for current year}~\\

{\bf Load balancing}\\

In this work we propose the improvement of the performance and scalability 
of parallel seismic wave models through dynamic load balancing. These models 
suffer from load imbalance for two reasons. First, they add a specific 
numerical condition at the borders of the domain, in order to absorb the 
outgoing energy. The decomposition of the domain into a grid of subdomains, 
which are distributed among tasks, creates load differences between the
tasks that simulate the borders and those responsible for the central 
subdomains. Second, the propagation of waves in the simulated area changes 
the workload on the subdomains on different time-steps. Therefore causing 
dynamic load imbalance. In order to evaluate the use of dynamic load balancing,
we ported a seismic wave simulator to Adaptive MPI, to benefit from its load 
balancing framework. Our experimental results show that dynamic load balancers 
can adapt to load variations during the application’s execution and improve 
performance by 36\%.

This work was presented in the PDP 2014 conference\cite{PDP2014}. An
extended version will be published in the International Journal of
High Performance Computing and Applications (\cite{IJHPCALB}). Laercio
Pilla described most of the load balancers in his
PhD\cite{PhDLaercio}.

\vspace{0.5cm}
{\bf Power consumption}\\

Power consumption is one of the main challenges to achieve Exascale
performance. Current research trends aim at overcoming power
consumption constraints using low-power processors. Although new
processors feature sensors that enable precise power measurements,
they provide different interfaces to collect data, making it difficult
to correlate performance with energy consumption. To overcome this
issue, we developed a platform-independent tool that collects power
and energy data from homogeneous and heterogeneous systems. Using this
tool, it provides a detailed comparison between a low-power processor
(ARM big.LITTLE) and a high performance processor (Intel Sandy
Bridge-EP) using all applications from the NAS parallel benchmarks and
a real-world soil irrigation simulator. The results show that the
average power demand of Intel Sandy Bridge-EP is within $12.6X$ to
$152.4X$ higher than ARM big.LITTLE, whereas its average energy
consumption is within $1.6X$ to $7.1x$ superior. Overall, ARM
big.LITTLE presented a better performance/energy trade-off when it
takes less than $9.2X$ the execution time of Intel Sandy Bridge-EP to
solve the same problem.

This work was published in \cite{IETCDT} and \cite{JPDC}. 

Large-scale simulation of seismic wave propagation
is an active research topic. Its high demand for processing power
makes it a good match for High Performance Computing (HPC).
Although we have observed a steady increase on the processing
capabilities of HPC platforms, their energy efficiency is still
lacking behind. In this work, we analyze the use of a low-power
manycore processor, the MPPA-256, for seismic wave propagation
simulations. First we look at its peculiar characteristics such as
limited amount of on-chip memory and describe the intricate
solution we brought forth to deal with this processor’s idiosyn-
crasies. Next, we compare the performance and energy efficiency
of seismic wave propagation on MPPA-256 to other common-
place platforms such as general-purpose processors and a GPU.
Finally, we wrap up with the conclusion that, even if MPPA-256
presents an increased software development complexity, it can
indeed be used as an energy efficient alternative to current HPC
platforms, resulting in up to 71\% and 81\% less energy than a
GPU and a general-purpose processor, respectively.

This work was presented at the SBAC PAD conference in Paris \cite{SBAC}.

\vspace{0.5cm}

{\bf Load Balancing and Power Saving}\\

In this work, we focus on reducing the energy consumption of
imbalanced applications through a combination of load balancing and
Dynamic Voltage and Frequency Scaling (DVFS). Our strategy employs an
Energy Daemon Tool to gather power information and a load balancing
module that benefits from the load balancing framework available in
the CHARM++ runtime system. We propose two variants of our
energy-aware load balancer (ENERGYLB) to save energy on imbalanced
workloads without considerably impacting the overall system
performance. The first one, called Fine- Grained EnergyLB
(FG-ENERGYLB), is suitable for plat- forms composed of few tens of
cores that allow per-core DVFS. The second one, called Coarse-Grained
EnergyLB (CG-ENERGLB) is suitable for current HPC platforms composed
of several multi-core processors that feature per-chip DVFS.

This work was presented at the HiPC conference \cite{HiPC}.

\paragraph{Visits and meetings}~\\

\begin{itemize}
\item Edson Padoin, November 2014, JLESC Workshop, Chicago
\item Jean-François Méhaut, November Z014, JLESC Workshop, Chicago
\item Brice Videau, June 2015, JLESC Workshop, Barcelona
\item Jean-François Méhaut, June 2015, JLESC Workshop, Barcelona
\end{itemize}


\paragraph{Impact and publications}~\\

% print list of publications containing the ``own'' keyword (for publications done within this project and year)
\printbibliography[heading=none,keyword=own]

\paragraph{Person-Month efforts}~\\

\begin{itemize}
\item Rafael Keller Tesser: 6 PMs
\item Edson Luiz Padoin: 4 PMs
\item Philippe Navaux: 1 PM
\item Celso Mendes:: 0.5 PM
\item Sanjay Kale: 0.25 PM
\item Jean-François Méhaut: 1 PM
\end{itemize}

\paragraph{Future plans}~\\

\begin{itemize}

\item Using simulations (SimGrid, BigSim, Dimemas) for the design and 
analysis of load balancers (Rafael Tesser, Philippe Navaux, Arnaud Legrand, Celso Mendes)

\item Load Balancing and heterogenous platforms/processors (Victor Martinez, 
Fabrice Dupros/BRGM, Philippe Navaux, Jean-François Méhaut)

\end{itemize}


\paragraph{References}~\\

Provide a few bibliographical references here.

% print list of publications not from this project but of other relevance
\printbibliography[heading=none,notkeyword=own]

\end{refsection}
