%===================================================================================================
\subsubsection{Exploiting Active Storage for Resilience}
\label{sec.report.gvr-bgas}
%===================================================================================================

\begin{refsection}[gvr-bgas]

%---------------------------------------------------------------------------------------------------
\paragraph{Participants}~\\
%---------------------------------------------------------------------------------------------------

\noindent
Professor Andrew A.~Chien, Argonne National Laboratory and University of Chicago \\
Prof.~Dr.~Dirk Pleiter, J\"{u}lich Supercomputing Centre \\
Dr.~Nan Dun, Argonne National Laboratory and University of Chicago \\
Nicolas Vandenbergen, J\"{u}lich Supercomputing Centre \\

%---------------------------------------------------------------------------------------------------
\paragraph{Research topic and goals}~\\
%---------------------------------------------------------------------------------------------------

The research in this topic is based on a hardware and a software architecture,
which both are currently under development: GVR and BGAS.
GVR (Global View Resilience) is a user-level library that enables
portable, efficient, application-controlled resilience \cite{Chien2015}.
It focusses on achieving scalability and maximization of error recovery.
BGAS (Blue Gene Active Storage) is a realisation of an active storage architecture based on
custom flash memory cards which are integrated into Blue\,Gene/Q I/O drawers.
Here JSC continues previous work on integration of non-volatile memory \cite{juniors}.
In this subproject our goal is to explore the opportunities of both architectures by
integrating them. More specifically the following research questions are addressed:
%
\begin{compactitem}
\item How well can the software architecture of GVR exploit the BGAS hardware architecture?
\item How efficiently can both architectures be exploited?
\item What is the value of active storage for a which classes of large-scale scientific computing?
\end{compactitem}

%---------------------------------------------------------------------------------------------------
\paragraph{Results for current year}~\\
%---------------------------------------------------------------------------------------------------

The project only started early this year and focussed on reaching the first milestone,
i.e.~completing the port of GVR to BGAS.
Based on a non-production version of MPI, namely MPICH3, this goal has been achieved
with both GVR running on the Blue\,Gene/Q system at JSC and the BGAS run-time system adapted
and recompiled for this version of MPI.

%---------------------------------------------------------------------------------------------------
\paragraph{Visits and meetings}~\\
%---------------------------------------------------------------------------------------------------

Beyond regular contacts via email the following meetings involving most of the participants
took place:
\begin{compactitem}
\item Meeting of A.~Chien, N.~Dun and D.~Pleiter at SC14 on November 17, 2014.
\item Technical update meeting on February 5, 2015.
\end{compactitem}

We plan for the following meetings:
\begin{compactitem}
\item Monthly telephone conferences starting from May 2015.
\item Technical update meeting at JLESC workshop, June/July 2015.
\item Architecture evaluation meeting at the end of July 2015.
\item Data transport analysis and technical update meeting during SC15, November 2015.
\end{compactitem}

%---------------------------------------------------------------------------------------------------
\paragraph{Impact and publications}~\\
%---------------------------------------------------------------------------------------------------

The project just started and thus did not result in any publications, yet.

% print list of publications containing the ``own'' keyword (for publications done within this project and year)
%\printbibliography[heading=none,keyword=own]
%
%Further impact
%\begin{itemize*}
%    \item done this
%    \item impact there
%\end{itemize*}

%---------------------------------------------------------------------------------------------------
\paragraph{Person-Month efforts}~\\
%---------------------------------------------------------------------------------------------------

%This is very important for the JLESC activity report. 
%Detail person-months spent by both permanent and temporary researchers 
%who worked for the collaboration.
The efforts of the participants are as follows:

\begin{center}
\begin{tabular}{|l|c|}
\hline
A.~Chien		& 0.10 PM \\
D.~Pleiter		& 0.10 PM \\
N.~Dun			& 2.00 PM \\
N.~Vandenbergen		& 0.50 PM \\
\hline
\end{tabular}
\end{center}

%---------------------------------------------------------------------------------------------------
\paragraph{Future plans}~\\
%---------------------------------------------------------------------------------------------------

Within this subproject the next steps will be:
%
\begin{compactitem}
\item Complete the already started performance evaluation and perform a scalability analysis.
\item Explore possible changes to both the GVR software architecture as well as the BGAS
      architecture (DSA interface to the flash memory, run-time system services).
\item Explore and evaluate different data transport mechanisms including optimized RDMA
      protocols for compute node to I/O node as well as I/O node to flash memory communication.
\end{compactitem}

%---------------------------------------------------------------------------------------------------
\paragraph{References}~\\
%---------------------------------------------------------------------------------------------------

% print list of publications not from this project but of other relevance
\printbibliography[heading=none,notkeyword=own]

\end{refsection}
