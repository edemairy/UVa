\documentclass[12pt]{article}
\usepackage[utf8]{inputenc}
\usepackage[american]{babel}
\usepackage{fontenc}
\usepackage{graphicx}
\usepackage{amsfonts,amsthm,amsmath,amssymb}
\usepackage{xspace}
\usepackage{todonotes}
\usepackage{fullpage}
\usepackage{paralist}
\usepackage{mdwlist}

\usepackage[backend=biber]{biblatex}

% Add your bib-file here, use report2015_xx.bib as filename with xx being your initials
\addbibresource{report2015_bjnw.bib}
\addbibresource{gvr-bgas.bib}


%collaboration symbols
\usepackage{color}
\newcommand{\ema}[1]{\ensuremath{#1}\xspace}
\newcommand{\ongoing}{{\Large \color{green}{\ema{\rightrightarrows}}\xspace}}
\newcommand{\starting}{{\Large \color{green}{\ema{| \hspace*{-0.3cm}\rightrightarrows}}\xspace}}
\newcommand{\finished}{{\Large \color{blue}{\ema{\rightrightarrows \hspace*{-0.3cm} |}}\xspace}}
%\newcommand{\blocked}{{\Large \color{red}{\ema{\nRightarrow}}\xspace}}
\newcommand{\blocked}{{\Large \color{red}{\ema{\downdownarrows}}\xspace}}
%\newcommand{\explore}{{\Large \color{black}{\ema{\rightrightarrows}}\xspace}}
\newcommand{\explore}{{\Large \color{magenta}{\ema{\upuparrows}}\xspace}}
%\theoremstyle{theorem}
\newtheorem{theorem}{Theorem}
\newtheorem{lemma}{Lemma}
\newtheorem{proposition}{Proposition}
\newtheorem{property}{Property}
\theoremstyle{definition}
\newtheorem{definition}{Definition}

\title{JLESC -- Activity Report 2015}

\author{Franck Cappello, William Kramer, Jesus Labarta, Yves Robert, Robert Speck}

\date{June 2015}


\begin{document}
\maketitle


\begin{abstract}
This is the 2015 activity report of the JLESC.
\end{abstract}


\section{Introduction}
\label{sec.intro}

\subsection{Purpose} 
The purpose of the Joint Laboratory for Extreme Scale Computing (JLESC) is to be an international, virtual organization whose goal is to enhance the ability of member organizations and investigators to make the bridge between Petascale and Extreme computing, following the model of the Joint Laboratory on Petascale Computing (JLPC). The founding partners of the JLESC are Inria and UIUC.  JLESC will also involve computer scientists, engineers and scientists from other disciplines and from industry, to ensure that the research facilitated by the Laboratory addresses science and engineering's most critical needs and takes advantage of the continuing evolution of computing technologies.

\subsection{Term and Termination} 
The JLESC will last four (4) years, from the date of signature of the agreement (June 2014). After the original time period, the agreement can be renewed if the majority of full partners agree by  written amendment. The Agreement may only be revised by written consent of the Founding Partners.
The Agreement may be terminated at any time by the Founding Partners mutual consent in writing. Alternatively, a Partner that wishes to terminate its participation in the JLESC will provide ninety (90) days' notice of such termination.

\subsection{Objectives} 
The objectives of the JLESC are to initiate and facilitate international collaborations on research and state of the practice topics, related to computational and data focused simulation and analytics at scale. The JLESC will facilitate the production of original ideas, publications, discussion forums, research reports, products and open source software, aimed to address the most critical issues in advancing from petascale to extreme scale computing.

\subsection{Research Topics, Activities and Joint Projects} 
JLESC research topics include: parallel programming models and libraries, numerical algorithms and libraries, parallel I/O systems and libraries, data analytics, graph algorithms, heterogeneous computing, resilience, system and application performance analysis and modeling, storage infrastructure design and efficiency and productivity tools.  Research topics may evolve over the course of the JLESC existence.
JLESC Joint Projects are defined and tracked by the JLESC.  A joint project will have at least two co-Principal Investigators, each from the different JLESC partner institutions involved in the project, a statement of goals for the project, a program of work, and a  schedule to accomplish the goals.  

The activities funded by JLESC will include:
\begin{compactenum}
\item A 2 to 3-days workshop every six months with all JLESC partners. The purpose of these workshops is to discuss the work performed in the joint projects during the last six months and to jointly plan work for the next six months. 
\item Researcher/student exchanges. These exchanges take the form of short and long term visits (from days to 1 year or more) between staff and students of JLESC institutions, at different JLESC partner sites, in order to work on joint projects. If required by host institution, separate agreements will be signed between the visiting and host institution specifying the terms of the exchange.
\end{compactenum}

It is expected the number of staff participating and the degree of contributions will be approximately the same across all partner institutions.

\subsection{Partnership}
JLESC will have three levels of partners: Founding Partners, Full Partners and Associate Partners.   
The Founding Partners are Inria and UIUC and are considered as initial Full Partners. 

Full Partners participate in the JLESC management, have the ability to attend workshops and JLESC functions; will pose and co-lead collaborative workshops, will commit certain levels of funding to the JLESC collaboration for specified time periods; will contribute to setting the priorities of JLESC; and will provide access to facilities at their institution (if available).  
Associate Partners are those entities who intend to become Full Partners but are not yet meeting the obligations of full partnership as described in this document.  
The maximum period an institution can be an Associate Partner is twelve months.  After that period, the institution either reaches full partnership or leaves the JLESC. The Founding Partners may agree to extend the period of associate partnership with good cause.
In order to retain active status as a JLESC partner, institutions must fully participate in JLESC activities. Any institution that has not fully and actively participated in at least two JLESC activities over an eighteen (18) month span will be considered for having its partnership revoked. 

The current members of the JLESC are:
\begin{compactenum}
\item UIUC/NCSA (Founding partner)
\item Inria (Founding partner)
\item ANL (Full partner)
\item BSC (Full partner)
\item JSC (Full partner)
\item Riken (Full partner)
\end{compactenum}

\subsection{Funding Model}
Each Full Partner institution will provide funding dedicated to support the two types JLESC activities described above. The mechanism is that each partner institution allocates an annual budget that is managed locally (for ease of administration) both by the local representative and the JLESC director.  Each Full Partner's contributions dedicated to JLESC activities will be approximately equal in value.  Collaboration effort costs, staff costs, or in-kind costs are not considered dedicated funding.  
Full Partnership will be executed by signing the statement in Attachment A and recording it with the Founding Partners.  Attachment A states the new Full Partner undertakes the rights and obligations of the agreement.  The new partner is considered a Full Partner when Attachment A is signed by an authorized representative of the new partner and the chairman of the JLESC steering committee.  
Due to the US Department of Energy Policy, the Founding Partners agree to a one time exception for Argonne National Laboratory (ANL). 

ANL's Full Partnership membership is based on a letter of commitment from Dr. Marc Snir to carry out all Full Partner responsibilities as an employee of ANL and as defined in this agreement, including equivalent funding of JLESC activities and responsibilities.
The level of JLESC funding from each Full Partner will be set by agreement of the Founding Partners.
Associate Partners shall allocate an annual budget that is managed locally and controlled both by the local representative and the JLESC director, but would be less than the funding levels provided by Full Members. 
Associate Partners may also directly cover the full cost of their staff participation in JLESC activities, including paying participation fees for attending JLESC activities. In the latter case, the levels of funding should be sufficient for equivalent participation in the JLESC activities.  Collaboration costs, staff costs, or in-kind costs are not considered in determining an equivalent value.  Associate Partnership will be indicated by providing a "letter of commitment" to participate in JLESC activities to the extent the partner adds observable value to the JLESC.
For visiting activities, whatever is the selected funding model, a researcher from institution A, visiting institution B will only be supported by funds provided by institutions A and B. 
Each party involved in the collaboration will cover its personnel salaries and associated costs. The home institution(s) of the director shall cover the director's salary.
There will be approximately one-year overlap of the JLPC agreement between NCSA and Inria and the JLESC agreement.  For the sake of this agreement, any remaining JLPC funding at the expiration of the JLPC agreement will transition to JLESC funding. 

\subsection{Intellectual Property}
Whenever and as often as possible, the partners will produce open publications and results, with the organizations participating in the collaborative effort jointly deciding on the intellectual property rights for the work carried out by the partners.  
For any results that require protection of Intellectual Property, the institutions participating in the development of that result will jointly resolve it between themselves by developing specific agreements for covering those activities and IP rights.

\subsection{Management and Governance} 
The Steering Committee will consist of the heads (or their representative) of the involved Full Partner institutions. The Steering Committee is responsible for the overall success of JLESC activities, for ensuring that the JLESC has sufficient participation from their respective institutions and for providing strategic guidance to the Director and Executive committee. 
The Steering Committee shall self-select a chairperson once a year. All decisions should be taken unanimously. If a unanimous decision cannot be reached, Founding Partners will determine the final decision amicably, in a spirit of cooperation and goodwill.
\newline
\newline
The current members of the JLESC steering committee are:
\begin{compactenum}
\item Ed. Siebel, Chair of the steering committee (UIUC/NCSA)
\item Antoine Petit (Inria)
\item William Harrod (DoE)
\item Mateo Valero (BSC)
\item Thomas Lippert (JSC)
\item Hirao Kimihiko (Riken)
\end{compactenum}
\newline
\newline
A Scientific Advisory Committee will assess the progress and the output of JLESC and suggest/recommend evolutions/adjustments on strategic and financial questions. The Scientific Advisory Committee will be composed of external researchers and experts of JLESC research topics and will provide evaluation, advice and recommendations to the JLESC Steering  Committee,  Director and Executive Committee in a written report. Scientific Advisory Committee members will be recruited from non-partner entities. 
\newline
\newline
The current members of the JLESC scientific Advisory committee are:
\begin{compactenum}
\item Miron Livny (U .Wisc), (USA)
\item Jack Dongarra (U. Tennessee), (USA)
\item Iain Duff (Rutherford Appleton Laboratory), (Europe)
\item Thomas Schulthess (CSCS), (Europe)
\item Satoshi Matsuoka (Titech) (Asia)
\item David Abramson (University of Queensland) (Asia)
\end{compactenum}
\newline
\newline
The JLESC will have a Director reporting to the Steering Committee every year. The Director will be appointed for a 2 years term by the Steering Committee. The JLESC Director is responsible for working with partner institutions to manage the JLESC activities on a day to day basis and manage the JLESC funds.  The JLESC Director will provide a yearly report of JLESC activities and accomplishments to the Steering and Scientific Advisory Committees.
\newline
\newline
The current JLESC director is Franck Cappello from ANL.
\newline
\newline
The JLESC Executive Committee will consist of one JLESC representative of each full partner institution. The members of the Executive Committee will be designated by the Steering Committee and the JLESC Director will be chosen among them. The JLESC representatives are responsible for overseeing their institutions' collaborative efforts and for assisting the Director in stimulating and organizing activities and in creating the annual report. 
The Executive Committee will meet periodically as necessary to carry out their responsibilities. The Steering, Scientific Advisory and Executive Committees will meet at least once every year during the annual evaluation days. 
\newline
\newline
The current members of the JLESC executive committee are:
\begin{compactenum}
\item Bill Kramer (UIUC/NCSA)
\item Yves Robert (Inria)
\item Franck Cappello (ANL)
\item Jesus Labarta (BSC)
\item Robert Speck (JSC)
\item Mitsuhisa Sato (Riken)
\end{compactenum}
\\
\subsection{Facilities} 
An institution involved in JLESC will provide to JLESC visitors involved in collaborations (if available): office space and internet access, administrative support, and make all reasonable efforts to provide them access to its local High Performance Computing resources during their visit.

\subsection{New Members} 
Organizations may apply for JLESC membership by contacting the Director and submitting a self-nomination letter explaining the value the submitting organization will bring to the JLESC and the organization's commitment to participate as described in this document.  New partners will be approved by unanimous agreement of the founding partners. 

\section{Scientific Activities}
\label{sec.science}

\subsubsection*{Activity reports}

The next pages present the scientific activities conducted between July 2014 and June 2015.
Scientific activities are presented as activity reports of on-going collaborative projects.
Each report features 8 parts: 
\begin{compactenum}
\item Participants
\item Research topic and goals
\item Results for current year
\item Visits and meetings (done and planned)
\item Impact and publications
\item Person-Month efforts
\item Future plans
\item References
\end{compactenum}

\subsubsection*{Scientific collaborations}

Scientific collaborations cover the three main topics of the joint-laboratory: parallel programming, resilience and numerical libraries. Other activities have been started on system issues and, in particular, related to file systems, archive and graph partitioning.
Scientific collaborations are categorized under five types:
\begin{compactitem}
\item Ongoing collaboration (\ongoing): Collaborative research is active, visits have been made, there have been several meetings and for some activities results have been produced and impact is significant.
\item Starting collaboration \starting): Collaborators from both sides have been identified and collaborative research has just started. Visits have been made or are planned. There is no result yet.
\item Finished collaboration (\finished): The topic behind the collaboration has been explored and produced some results. Collaborators decided to continue in other directions. 
\item Blocked/stopped collaboration (\blocked): Such collaboration has not progressed over the past year some for reasons associated with the availability of software, data or people. Ultimately a blocked collaboration get stopped if no progress can be made.
\item Collaboration under exploration (\explore): Collaboration is not yet established. However activity has started and gave some impact. We put in this category, activities that are likely to become collaborative research.
\end{compactitem}

\subsection{Resilience}

\subsubsection{New Techniques to Design Silent Data Corruption Detectors}
\label{Cappello-SDC-detectors} % add label, replace xx with your initials

% start new section for referencing, please use the name of your bibfile here instead of report2015 (.bib)
\begin{refsection}[collab_report_FC_SDC-detectors]

\paragraph{Participants}

\begin{itemize*}
 \item Omer Subasi, BSC, Ph. D. Student 
 \item Leonardo Bautista Gomer, ANL, Postdoc 
 \item Sheng Di, ANL, Postdoc 
 \item Franck Cappello, ANL, Senior Computer Scientist
 \item Jesus Labarta, BSC, Director of Computer Science Department
 \item Osman Unsal, BSC, Senior Researcher 
 \end{itemize*}

\paragraph{Research topic and goals}~\\

Supercomputers allow scientists to study natural phenomena by means of computer simulations. Next-generation machines are expected to have more components and, at the same time, consume several times less energy per operation. These trends are pushing supercomputer construction to the limits of miniaturization and energy-saving strategies. Consequently, the number of soft errors is expected to increase dramatically in the coming years. While mechanisms are in place to correct or at least detect some soft errors, a significant percentage of those errors pass unnoticed by the hardware. Such silent errors are extremely damaging because
they can make applications silently produce wrong results. In this project we propose multiple corruption detectors. A first type of detectors leverages certain properties of high-performance computing applications in order to detect silent errors at the application level. This technique detects corruption solely based on the behavior of the application datasets and is application-agnostic. We propose and we couple them to work together in a fashion transparent to the user. A second type of detectors is based on replication at task level.

\paragraph{Results for current year}~\\

This collaboration is starting. The two teams have a strong record of publications in this domain ~\cite{Leo-PPoPP2014,Edu-HPDC2015,Di-CCGRID2015,Edu-SC2014}.
The collaboration will be implemented by the 1 week visit of Franck Cappello to BSC and the 3 months visit of Omer Subasi at Argonne from mid July 2015. 

\paragraph{Visits and meetings}

\begin{itemize*}
 \item Franck Cappello (ANL) to BSC from July 6th to July 10th, 2015.
 \item Omer Subasi (BSC) to ANL from July 15th to October 15, 2015.
\end{itemize*}


\paragraph{Impact and publications}~\\

% print list of publications containing the ``own'' keyword (for publications done within this project and year)
\printbibliography[heading=none,keyword=own]

Further impact
\begin{itemize*}
    \item none yet
% \item impact there
\end{itemize*}

\paragraph{Person-Month efforts}~\\

Efforts will be spent from July 2015.

\paragraph{Future plans}~\\

After the two visits planned from July 2015, we plan to work on a join-paper, submit if to IPDPS, PPoPP or CCGRID and then present the results during the November JLESC workshop.

\paragraph{References}~\\

Provide a few bibliographical references here.

% print list of publications not from this project but of other relevance
\printbibliography[heading=none,notkeyword=own]

\end{refsection}

%===================================================================================================
\subsubsection{Exploiting Active Storage for Resilience}
\label{sec.report.gvr-bgas}
%===================================================================================================

\begin{refsection}[gvr-bgas]

%---------------------------------------------------------------------------------------------------
\paragraph{Participants}~\\
%---------------------------------------------------------------------------------------------------

\noindent
Professor Andrew A.~Chien, Argonne National Laboratory and University of Chicago \\
Prof.~Dr.~Dirk Pleiter, J\"{u}lich Supercomputing Centre \\
Dr.~Nan Dun, Argonne National Laboratory and University of Chicago \\
Nicolas Vandenbergen, J\"{u}lich Supercomputing Centre \\

%---------------------------------------------------------------------------------------------------
\paragraph{Research topic and goals}~\\
%---------------------------------------------------------------------------------------------------

The research in this topic is based on a hardware and a software architecture,
which both are currently under development: GVR and BGAS.
GVR (Global View Resilience) is a user-level library that enables
portable, efficient, application-controlled resilience \cite{Chien2015}.
It focusses on achieving scalability and maximization of error recovery.
BGAS (Blue Gene Active Storage) is a realisation of an active storage architecture based on
custom flash memory cards which are integrated into Blue\,Gene/Q I/O drawers.
Here JSC continues previous work on integration of non-volatile memory \cite{juniors}.
In this subproject our goal is to explore the opportunities of both architectures by
integrating them. More specifically the following research questions are addressed:
%
\begin{compactitem}
\item How well can the software architecture of GVR exploit the BGAS hardware architecture?
\item How efficiently can both architectures be exploited?
\item What is the value of active storage for a which classes of large-scale scientific computing?
\end{compactitem}

%---------------------------------------------------------------------------------------------------
\paragraph{Results for current year}~\\
%---------------------------------------------------------------------------------------------------

The project only started early this year and focussed on reaching the first milestone,
i.e.~completing the port of GVR to BGAS.
Based on a non-production version of MPI, namely MPICH3, this goal has been achieved
with both GVR running on the Blue\,Gene/Q system at JSC and the BGAS run-time system adapted
and recompiled for this version of MPI.

%---------------------------------------------------------------------------------------------------
\paragraph{Visits and meetings}~\\
%---------------------------------------------------------------------------------------------------

Beyond regular contacts via email the following meetings involving most of the participants
took place:
\begin{compactitem}
\item Meeting of A.~Chien, N.~Dun and D.~Pleiter at SC14 on November 17, 2014.
\item Technical update meeting on February 5, 2015.
\end{compactitem}

We plan for the following meetings:
\begin{compactitem}
\item Monthly telephone conferences starting from May 2015.
\item Technical update meeting at JLESC workshop, June/July 2015.
\item Architecture evaluation meeting at the end of July 2015.
\item Data transport analysis and technical update meeting during SC15, November 2015.
\end{compactitem}

%---------------------------------------------------------------------------------------------------
\paragraph{Impact and publications}~\\
%---------------------------------------------------------------------------------------------------

The project just started and thus did not result in any publications, yet.

% print list of publications containing the ``own'' keyword (for publications done within this project and year)
%\printbibliography[heading=none,keyword=own]
%
%Further impact
%\begin{itemize*}
%    \item done this
%    \item impact there
%\end{itemize*}

%---------------------------------------------------------------------------------------------------
\paragraph{Person-Month efforts}~\\
%---------------------------------------------------------------------------------------------------

%This is very important for the JLESC activity report. 
%Detail person-months spent by both permanent and temporary researchers 
%who worked for the collaboration.
The efforts of the participants are as follows:

\begin{center}
\begin{tabular}{|l|c|}
\hline
A.~Chien		& 0.10 PM \\
D.~Pleiter		& 0.10 PM \\
N.~Dun			& 2.00 PM \\
N.~Vandenbergen		& 0.50 PM \\
\hline
\end{tabular}
\end{center}

%---------------------------------------------------------------------------------------------------
\paragraph{Future plans}~\\
%---------------------------------------------------------------------------------------------------

Within this subproject the next steps will be:
%
\begin{compactitem}
\item Complete the already started performance evaluation and perform a scalability analysis.
\item Explore possible changes to both the GVR software architecture as well as the BGAS
      architecture (DSA interface to the flash memory, run-time system services).
\item Explore and evaluate different data transport mechanisms including optimized RDMA
      protocols for compute node to I/O node as well as I/O node to flash memory communication.
\end{compactitem}

%---------------------------------------------------------------------------------------------------
\paragraph{References}~\\
%---------------------------------------------------------------------------------------------------

% print list of publications not from this project but of other relevance
\printbibliography[heading=none,notkeyword=own]

\end{refsection}


\subsubsection{Algorithms for detecting and correcting silent errors}
\label{sec.report.silent} % add label, replace xx with your initials

% start new section for referencing, please use the name of your bibfile here instead of report2015 (.bib)
\begin{refsection}[silent-errors]
\nocite{*}

\paragraph{Participants}~\\

\begin{itemize}
	\item Anne Benoit (INRIA)
	\item Aur\'{e}lien Cavelan (INRIA)
	\item Leonardo Bautisat Gomez (ANL)
	\item Franck Cappello (ANL)
	\item Sheng Di (ANL)
	\item Yves Robert (INRIA)
	\item Hongyang Sun (INRIA)
	\item Fr\'{e}d\'{e}ric~Vivien (INRIA)
\end{itemize}	

\paragraph{Research topic and goals}~\\

The goal of this research is to devise effective resilience protocols to detect and correct silent errors during a computation, in order to optimize execution time or energy consumption. The studied topics include:

\begin{itemize}
\item Finding optimal patterns in the checkpoint and rollback recovery technique using guaranteed and partial verifications for error detection (extending the work of\~cite{young74,daly04} for fail-stop failures)
\item Devising optimal voltage overscaling algorithms to cope with timing errors for matrix operations with the help of algorithm-based fault tolerance (ABFT), extending the work of\~cite{WapcoQuintan}
to timing errors
\end{itemize}

\paragraph{Results for current year}~\\

The publications produced in this research are \cite{Benoit14_chain, Cavelan15_partialSaurabh, Cavelan15_timing,Gomez14_SDCDetector}.

In \cite{Benoit14_chain}, we combine the traditional checkpointing and rollback recovery strategies with guaranteed verification mechanisms to address both fail-stop and silent errors. The objective is to minimize either makespan or energy consumption. While DVFS is a popular approach for reducing the energy consumption, using lower speeds/voltages can increase the number of errors, thereby complicating the problem. We consider an application workflow whose dependence graph is a chain of tasks, and we study three execution scenarios: (i) a single speed is used during the whole execution; (ii) a second, possibly higher speed is used for any potential re-execution; (iii) different pairs of speeds can be used throughout the execution. For each scenario, we determine the optimal checkpointing and verification locations (and the optimal speeds for the third scenario) to minimize either objective. For both objectives, we also derive the optimal checkpointing and verification periods for the divisible load application model, where checkpoints and verifications can be placed at any point in execution of the application. The different execution scenarios are assessed and compared through an extensive set of experiments.

In~\cite{Cavelan15_partialSaurabh}, we investigate the use of partial verification mechanisms in addition to a guaranteed verification for detecting silent errors. The main objective is to investigate to which extent it is worthwhile to use some light-cost but less precise verification type in the middle of a periodic computing pattern, which ends with a guaranteed verification right before each checkpoint. Introducing partial verifications dramatically complicates the analysis, but we are able to analytically determine the optimal computing pattern (up to first-order approximation), including the optimal length of the pattern, the optimal number of partial verifications, as well as their optimal positions inside the pattern.
Simulations based on a wide range of parameters confirm the benefits of partial verifications in certain scenarios, when compared to the baseline algorithm that uses only guaranteed verifications.

In \cite{Cavelan15_timing}, we propose a software-based approach using dynamic voltage 
overscaling~\cite{Ciocca04_ALFT}
to reduce the energy consumption of HPC applications. This technique
aggressively lowers the supply voltage below nominal voltage, which
introduces timing errors. These errors occur when the results of some logic gates are used before their output signals
reach their final values, thus causing silent data corruption. To cope with timing errors, we use Algorithm-Based Fault-Tolerance
(ABFT) to provide fault tolerance for matrix operations. We introduce
a formal model, and we design optimal polynomial-time solutions, to execute a linear chain of tasks. Evaluation results obtained for matrix multiplication demonstrate that our approach indeed leads to significant energy savings, compared to the standard algorithm that always operates at nominal voltage.

\paragraph{Visits and meetings}~\\

The collaboration is just starting. Leonarod Bautista Gomez has visited
ENS Lyon three days, from March31 to April 2, 2015.

%\begin{itemize}
%\item (Done) Workshop on Performance Modeling, Benchmarking and Simulation of High Performance Computer Systems (PMBS), Supercomputing (SC), Nov, 2014
%
%\item (Planned) Fault Tolerance for HPC at eXtreme Scale (FTXS) Workshop, ACM Symposium on High-Performance Parallel and Distributed Computing (HPDC), Jun, 2015
%
%\item (Planned) ICPP, others??
%\item ...
%\end{itemize}

\paragraph{Impact and publications}~\\

No common publication has been achieved yet, but each partner had contributed independent work listed 
below.\\

% print list of publications containing the ``own'' keyword (for publications done within this project and year)
\printbibliography[heading=none,keyword=own]



\paragraph{Person-Month efforts}~\\

%This is very important for the JLESC activity report.
%Detail person-months spent by both permanent and temporary researchers
%who worked for the collaboration.

\begin{itemize}
	\item {\bf Aur\'elien Cavelan}, temporary researcher, 8 months
	\item {\bf Hongyang Sun}, temporary researcher, 8 months
	\item {\bf Anne Benoit}, permanent researcher, 1 month
	\item {\bf Yves Robert}, permanent researcher, 4 months
	\item {\bf Fr\'{e}d\'{e}ric~Vivien}, permanent researcher, 2 months
\end{itemize}

\paragraph{Future plans}~\\

The future plans include extending this research from linear chains to other application workflows, such as tree graphs, fork-join graphs, series-parallel graphs, or even general directed acyclic graphs (DAGs). For the voltage overscaling problem, we are currently engaged in the design of optimal algorithms for a set of independent tasks.

For silent error detection, we will investigate the use of multiple partial verification with different costs and accuracies. The question is whether one can achieve better performance by utilizing more than one type of partial verification simultaneously. Another direction is to consider ``false positives", which we have omitted so far. Indeed, in many verification mechanism, a tradeoff exists between false positives and false negatives by adjusting the range parameters of the fault filters. Analyzing the performance in this situation is a challenging future direction.

\paragraph{References}~\\

% print list of publications not from this project but of other relevance
\printbibliography[heading=none,notkeyword=own]

\end{refsection}


\subsubsection{Hybrid resilience for MPI + tasks codes}
\label{Cappello-MPI-Tasks-Resilience} % add label, replace xx with your initials

% start new section for referencing, please use the name of your bibfile here instead of report2015 (.bib)
\begin{refsection}[collab_report_FC_MPI_task_resilience]

\paragraph{Participants}

\begin{itemize*}
 \item Omer Subasi, BSC, Ph. D. Student 
 \item Tatiana Martsinkevich, Inria, Ph. D. Student 
 \item Franck Cappello, ANL, Senior Computer Scientist
 \item Osman Unsal, BSC, Senior Researcher 
 \end{itemize*}

\paragraph{Research topic and goals}~\\

We investigate fault-tolerant protocols, mitigating transient errors, that can be applied to task-parallel message-passing applications. 
Our approach limits the consequences of a fault to the task that experienced it and allows a fast and asynchronous recovery that is more efficient than the
conventional full-application rollback-recovery. The proposed protocol will implemented in Nanos; a dataflow runtime for the task-based OmpSs programming model; and the PMPI profiling layer to fully support hybrid OmpSs+MPI applications.
The OmpSs+MPI programming model provides several advantages over conventional programming models in terms of performance and the potential for using asynchronous fault tolerance. 

\paragraph{Results for current year}~\\

This collaboration started in summer 2014.  In our experiments we demonstrate that our fault tolerant solution has a reasonable overhead, with a maximum observed overhead of 4.5\%. We also show that fine-grained parallelization is important for hiding the overheads related to the protocol as well as the recovery of tasks.

The collaboration has been implemented by the 3 months visit of  Tatiana Martsinkevich to BSC during the summer of  2014.
Since then a paper has been written and is under submission ~\cite{Martsinkevich-Cluster2015}.  

\paragraph{Visits and meetings}

\begin{itemize*}
 \item Tatiana Martsinkevich (Inria) to BSC from June 2014 to October, 2014.
\end{itemize*}


\paragraph{Impact and publications}~\\

A paper is under submission ~\cite{Martsinkevich-Cluster2015}.  
% print list of publications containing the ``own'' keyword (for publications done within this project and year)
\printbibliography[heading=none,keyword=own]

Further impact
\begin{itemize*}
    \item none yet
% \item impact there
\end{itemize*}

\paragraph{Person-Month efforts}~\\

\begin{itemize*}
 \item Tatiana Martsinkevich (Inria): 6 PM
 \item Omer Subasi (BSC): 4 PM 
 \item Osman Unsal (BSC): 0.5 PM 
 \item Franck Cappello (ANL): 1 PM 
\end{itemize*}

\paragraph{Future plans}~\\

After the two visits planned from July 2015, we plan to work on a join-paper, submit if to IPDPS, PPoPP or CCGRID and then present the results during the November JLESC workshop.

\paragraph{References}~\\

Provide a few bibliographical references here.

% print list of publications not from this project but of other relevance
\printbibliography[heading=none,notkeyword=own]

\end{refsection}

\subsubsection{Programming Model Extensions for Resilience}
\label{sec.report.pb3} % add label, replace xx with your initials

\begin{refsection}[report2015_resilience]

\paragraph{Participants}~\\

\begin{itemize}
\item Abdelhalim Amer, Argonne National Laboratory, Postdoctoral Appointee
\item Pavan Balaji, Argonne National Laboratory, Computer Scientist
\item Vicenc Beltran, Barcelona Supercomputing Center
\item Marc Casas, Barcelona Supercomputing Center 
\end{itemize}

\paragraph{Research topic and goals}~\\

The reliability of high performance computing systems is predicted to worsen
given current technology trends, where several errors per day are expected to
occur in exascale systems. Consequently, the software stack that target HPC
systems needs to tolerate faults in order to cope with the instability of such
systems. Particularly, current parallel programming models, which assume that
applications will execute on failure-free systems, require resilience extensions.

The most popular models used to program parallel systems are MPI for distributed
memory and thread-based models, such as OpenMP, for shared-memory. Several works
have studied fault-tolerance methods for MPI applications at the process level
such as the User-Level Fault Mitigation (ULFM) effort at ANL. Recently, the
need for finer-grained levels of resilience arose in the community because of
the space and time overheads of coarse-grained approaches. Consequently,
research in this direction has become more active leading to new methods
such as the rollback-recovery method applied to asynchronous tasks in the OmpSs
programming model.

Despite recent advances, there exists no programming model that offers high
degrees of resilience and low overheads for hybrid MPI+threads applications.
This collaboration aims at providing resilience extensions for hybrid
MPI+threads programming models at multiple levels. At the thread-level, errors
are contained within the failing threads except if recovery is not possible, in
which case propagating the failure to the MPI process will be necessary. The
system then will rely on process-level fault-tolerance techniques to mitigate
the failure.

\paragraph{Results for current year}~\\

There were interactions through telecon between the collaborators which involved
discussions about ULFM for process-level resilience and the feasibility of
thread-level fault-tolerance extensions.

\paragraph{Visits and meetings}~\\

There were frequent telecon meetings since the beginning of he collaboration.
Planned visits: there are no planned visits yet.

\paragraph{Impact and publications}~\\

% print list of publications containing the ``own'' keyword (for publications
% done within this project and year)
\printbibliography[heading=none,keyword=own]

Further impact \begin{itemize*} \item done this \item impact there
\end{itemize*}

\paragraph{Person-Month efforts}~\\

Here we summarize the person-month efforts of the members involved in this
project since the beginning of the collaboration.

\begin{itemize}
\item Abdelhalim Amer 0.5
\item Pavan Balaji 1
\item Vicenc Beltran 1
\item Marc Casas 1
\end{itemize}

\paragraph{Future plans}~\\

We plan to push further the investigation of fine-grained fault-tolerance
support in hybrid MPI+threads programming models. The goal is to answer several
open questions including, but not limited to, the degree of granularity suitable for
the fine-grained level (threads or tasks), define the interaction between the
coarse-grained and fine-grained levels of fault-tolerance, the way communication issued by a
failed thread should be handled (flushed or ignored), and investigate the role and necessity of using
robust synchronization mechanisms, such as robust mutexes, by the underlying
multithreaded runtime system.

\paragraph{References}~\\

%Provide a few bibliographical references here.

% print list of publications not from this project but of other relevance
\printbibliography[heading=none,notkeyword=own]

\end{refsection}


\subsection{I/O, storage and visualization}

\subsubsection{Smart In Situ Visualization}
\label{sec.report.tp} % add label, replace xx with your initials

% start new section for referencing, please use the name of your bibfile here instead of report2015 (.bib)
\begin{refsection}[report2015_tp]

\paragraph{Participants}~\\

\begin{itemize*}
\item Lokman Rahmani, ENS Rennes, PhD student
\item Hadrien Croubois, ENS Lyon, Visiting scholar
\item Matthieu Dorier, ANL, Postdoc
\item Matthieu Dreher, ANL, Postdoc
\item Bruno Raffin, Inria, Senior Researcher
\item Tom Peterka, ANL, Senior Researcher
\item Luc Bougé, ENS Rennes, Professor
\item Gabriel Antoniu, Inria, Senior Researcher
\item Roberto Sisneros, NCSA, Senior Researcher
\item Justin Wozniak, ANL, Senior Researcher
\end{itemize*}

\paragraph{Research topic and goals}~\\

The increasing gap between computational power and I/O performance in new supercomputers drives
a shift from an offline approach of data analysis to an inline approach, termed in situ
visualization (ISV). While many parallel visualization tools now provide ISV, the trend has
been to feed such software with what previously was large dumps of raw data, and let them
render everything at the highest possible resolution. This leads to a potentially large
performance impact in simulations that support ISV, in particular when ISV is performed
interactively. We are researching smarter methods of performing ISV. Our approach aims
to detect potentially interesting regions in the generated dataset in order to feed ISV
frameworks with only a subset of the data produced by the simulation. While this method
mitigates the load on ISV frameworks, making them more efficient and more interactive, it also
helps scientists focus on the relevant part of their data.

\paragraph{Results for current year}~\\

We are developing a smart ISV framework that allows for data reduction based on its
potential scientific value, controlled by the user and transparent to the simulation and the
analysis and visualization tools. We implemented such a solution for data generated with a spatial
decomposition of the simulation domain across processes. We choose to define the scientific
value of the filtered data as the quality of visualization (QoV) of the images generated by
rendering it, and we use the structural similarity index metric (SSIM)~\cite{ssim} to objectively
quantify QoV loss. SSIM is a reference metric to quantify the human perception of loss quality
in a compressed image. It considers the structural differences of an image with respect to a
reference one.

Data reduction in our contribution is based on a generic, simple yet efficient estimation of
relevant data in terms of local, spatial variability. We are investigating different metrics
based on statistics, information theory and image processing, to compute this local
variability. The resolution at which a region of data is rendered then depends on the local
value of these metrics.  We evaluated Smart ISV through experiments on the French Grid’5000
using the CM1 atmospheric simulation~\cite{Bryan:2002}, with VisIt as a visualization backend
and Damaris/Viz as the interface between CM1 and VisIt. Damaris/Viz allows us to define our
data filters as plugins, thus making the “Smart” component of ISV transparent to both CM1 and
VisIt.  Results show an up to 40\% gain in rendering performance with limited loss in QoV, when
using our data reduction algorithms.

\paragraph{Visits and meetings}~\\

Past visits:
\begin{itemize*}
\item October - December 2014: Lokman Rahmani 3 month - visit to ANL
\item July 2014 - June 2015: Hadrien Croubois 12 month visit to ANL
\end{itemize*}

Planned visits:
\begin{itemize*}
\item June 2015: will meet with Gabriel Antoniu, Bruno Raffin, Justin Wozniak, Matthieu Dorier,
  Matthieu Dreher in Barcelona to discuss progress of Smart Visualization and plan next project
  in Decaf-Swift integration.
\item September 2015: will meet with Gabriel Antoniu at Cluster 2015 in Chicago, IL to update
  progress
\item November 2015: will meet with Gabriel Antoniu and Roberto Sisneros at SC15 in Austin, TX to plan next steps
\item September 2015 - August 2016: extended invitation to Pierre-Louis Guhur (ENS Cachan) for
  12 month visit to ANL
\end{itemize*}

\paragraph{Impact and publications}~\\

We are preparing a manuscript for an upcoming systems or visualization
conference~\cite{rahmani15}.

\printbibliography[heading=none,keyword=own]

Further impact:

\begin{itemize*}
\item An approach for Smart ISV using iteration-level data reduction based on the scientific
  value of the data;
\item A domain-independent approach for estimating the scientific value of a dataset;
\item A comparative discussion of several metrics for relevant data selection, allowing to
  implement the aforementioned approach;
\item A prototype implementation of the proposed data selection and reduction algorithms into
  the Damaris/Viz ISV framework as an external plugin.
\item An evaluation showing a positive impact of our approach to smart ISV on the performance
  with little loss in quality of visualization, using a real atmospheric simulation.
\end{itemize*}

\paragraph{Person-Month efforts}~\\

\begin{itemize*}
\item Lokman Rahmani: 6 months
\item Matthieu Dorier: 1 month
\item Tom Peterka: 1 month
\item Luc Bougé: 0.5 month
\item Gabriel Antoniu: 0.5 month
\item Roberto Sisneros: 0.5 month
\end{itemize*}

\paragraph{Future plans}~\\

As of today, experiments were run with climate simulations only. Also, scientific relevance of
data was measured by their perception by human visual system. The smart ISV capabilities was
integrated in Damaris/Viz only but can be extended to any other ISV frameworks with plugin
management system and internal data model. We are considering extending the data
reduction process to other type of simulation, not based on space-partitioned datasets. An
advanced study of alternative selection metrics can be also interesting. Finally, to take full
profit of Damaris/Viz we can provide a VisIt plugin that interact dynamically with the data
reduction plugin at simulation runtime.

\paragraph{References}~\\

% print list of publications not from this project but of other relevance
\printbibliography[heading=none,notkeyword=own]

\end{refsection}


\subsubsection{In Situ Visualization of HPC Simulations using Dedicated Cores}
\label{sec.report.GA1} % add label, replace xx with your initials

% start new section for referencing, please use the name of your bibfile here instead of report2015 (.bib)
\begin{refsection}[damaris]

\paragraph{Participants}~\\

Matthieu Dorier, Lokman Rahmani, Tom Peterka, Gabriel Antoniu, Roberto Sisneros

\paragraph{Research topic and goals}~\\

Large scale simulations running on leadership class supercomputers generate massive amounts of data for subsequent analysis and visualization. As the performance of storage systems shows its limits, an alternative consists of embedding visualization and analysis algorithms within the simulation (in situ visualization). Our goal within this context is to explore the potential benefit of using Damaris (\url{http://damaris.gforge.inria.fr/}), a middleware for I/O forwarding and post-processing using dedicated cores, to offload in situ visualization while sharing resources with the running simulation.


\paragraph{Results for current year}~\\

%Write a few paragraphs on main results for current year. Describe relevant publications here~\cite{j132,j133}.

%\paragraph{Activity for 2014}

\subparagraph{Sub-goals:} In situ visualization using Damaris/Viz still poses the problem of large amounts of data requiring to be processed at a high rate, while only a small part of the data might be of interest for the end user. Our goal is to improve Damaris/Viz by enabling an automated detection of interesting features in the datasets in order to reduce the visualization payload and thus, further improve the performance of in situ visualization.

\subparagraph{Results:} To meet the above goal, we made use of ITL (Information Theory Library)~\cite{chaudhuri_ldav12} and DIY~\cite{peterka_ldav11} to help integrate the efficient computation of information theory based metrics. Implementation and evaluation are in progress on Grid'5000 and Blue Waters with various simulations including the CM1 tornado simulation~\cite{bryan2002benchmark}. A joint paper is in preparation on this topic.

\paragraph{Visits and meetings}~\\

\begin{itemize}
\item June 2 - June 6: Rob Ross visited KerData in Rennes.
\item June 9 - June 11: 11th workshop of the JLESC held in Nice, France.
\item September 30 - December 23: Lokman Rahmani did an internship at Argonne National Laboratory.
\item November 24 - November 26: Meetings for updates and planning were held during the 2nd JLESC workshop in Chicago.
\end{itemize}

\paragraph{Impact and publications}~\\

% print list of publications containing the ``own'' keyword (for publications done within this project and year)
\printbibliography[heading=none,keyword=own]

\begin{itemize*}
   \item  Demo by Matthieu Dorier at the Inria booth at SC'14.
    \item Matthieu Dorier succesfully defended his PhD thesis, which includes several lines of work developed in the framework of the collaboration with JLESC. Damaris is at the core of his PhD work. For all his results, Matthieu was awarded the $2^{nd}$ Award of Rennes 1 Foundation for the Matisse Doctoral school.
    \item An ADT proposal to fund a 2-year engioneer position for maintaining and improving Damaris has just been approved by Inria (2015-2017).
    \item Further evolutions of Damaris will take place during the postdoc fellowship of Matthieu Dorier at Argonne National Lab, started in February 2015.
\end{itemize*}

\paragraph{Person-Month efforts}~\\

%This is very important for the JLESC activity report. 
%Detail person-months spent by both permanent and temporary researchers 
%who worked for the collaboration.

\begin{itemize}
\item Gabriel Antoniu, Inria 1MM
\item Matthieu Dorier, ENS Rennes 5MM
\item Lokman Rahmani, ENS Rennes 4MM
\item Roberto Sisneros, NCSA 1MM
\item Tom Peterka, ANL 1MM
\item Rob Ross, ANL 1MM
\end{itemize}

\paragraph{Future plans}~\\

In situ approaches provide faster insight from the simulation and enable interactivity. They can be implemented using main memory (tight coupling between simulation and post-processing) or secondary storage (loose coupling). Both of the coupling modes have pros and cons, and neither is suitable for extreme-scale science. Tight coupling have the advantage of greater performance and less data moving, however, it is not resilient and the coupling is rigid. Loose coupling provides better resilience and more generic usage. We are looking for a solution that takes advantages of both of the coupling modes, while considering different criteria. The goal is to propose a generic, reusable and efficient platform for post-processing tasks scheduling and placement. Many optimization policies will be designed considering, mainly, storage throughput, data movement, time to solution and resilience.

\paragraph{References}~\\

% print list of publications not from this project but of other relevance
\printbibliography[heading=none,notkeyword=own]

\end{refsection}

\subsubsection{Mitigating I/O Interference in Concurrent HPC Applications}
\label{sec.report.GA2} % add label, replace xx with your initials

% start new section for referencing, please use the name of your bibfile here instead of report2015 (.bib)
\begin{refsection}[omniscio]

\paragraph{Participants}~\\

Matthieu Dorier, Gabriel Antoniu, Shadi Ibrahim, Rob Ross, Orcun Yildiz

\paragraph{Research topic and goals}~\\

With million-core supercomputers comes the problem of interference between distinct applications accessing a shared file system in a concurrent manner~\cite{lofstead2010managing}. Our work aims to investigate and quantify this interference effect, and to mitigate I/O interference through a novel approach that uses cross-application communication and coordination: CALCioM. In previous work experiments done during Matthieu Dorier's internship at ANL led to a better understanding of the I/O interference phenomena, and to the implementation of a prototype of the CALCioM approach with currently includes 3 scheduling strategies. As a result of this work, a paper was accepted at IEEE IPDPS 2014~\cite{calciom}. 


\paragraph{Results for current year}~\\

\subparagraph{Sub-goal 1:} Having exemplified the interference phenomenon on synthetic benchmarks, we are now interested in showing how often such interference occurs and the nature of the applications that are involved in this phenomenon. This investigation was done through the analysis of traces produced by the Darshan library on ANL's Intrepid BlueGene/P system.

\subparagraph{Results:} We developed Darshan-Ruby and Darshan-Web (\url{http://darshan-ruby.gforge.inria.fr}). Darshan-Ruby is a Ruby wrapper to ANL's Darshan library. Darshan-Web is a Web platform for online analysis of Darshan log files. This platform is based on Ruby on Rails, D3.js, and AJAX technologies. A demo is available here: \url{http://darshan-web.irisa.fr}

\subparagraph{Sub-goal 2:} Our second goal was to find a way to improve CALCioM by modeling and predicting I/O patterns. This prediction should be made at run time, with no prior knowledge of the application, and should converge toward an accurate model of the application's I/O within a few iterations only.

\subparagraph{Results:} To this end, we developed Omnisc'IO, an approach that leverages format grammars to model and predict the I/O behavior of HPC applications. Omnisc'IO was evaluated with four real application: CM1~\cite{bryan2002benchmark}, Nek5000~\cite{nek5000}, LAMMPS~\cite{lammps} and GTC~\cite{gtc}, and our results led to a paper at SC'14~\cite{omniscio}.

\paragraph{Visits and meetings}~\\

\begin{itemize}
\item June 2 - June 6: Rob Ross visited KerData in Rennes.
\item June 9 - June 11: 11th workshop of the JLESC held in Nice, France.
\item November 24 - November 26: Meetings for updates and planning were held during the 2nd JLESC 
\item July - September 2015: Internship of Or\c{c}un Yildiz at Argonne National Laboratory.
\end{itemize}

\paragraph{Impact and publications}~\\

% print list of publications containing the ``own'' keyword (for publications done within this project and year)
\printbibliography[heading=none,keyword=own]

%Further impact
%\begin{itemize*}
 %   \item Paper presented at SC'14.
%\end{itemize*}

\paragraph{Person-Month efforts}~\\


\begin{itemize}
\item Gabriel Antoniu, Inria 1MM
\item Shadi Ibrahim, Inria 2MM
\item Matthieu Dorier, ENS Rennes 3MM
\item Orçun Yildiz, ENS Rennes 10MM
\item Rob Ross, ANL 1MM
\end{itemize}

\paragraph{Future plans}~\\

\subparagraph{Sub-goal 1:} Our plan is to integrate Omnisc'IO within CALCioM in order to provide a fully-featured I/O scheduling system. Evaluations will be made using event-driven simulations, using traces from Argonne's Darshan tools.

\subparagraph{Sub-goal 2:} Our plan is to investigate the different factors which contribute to the interference in HPC systems and explore a model to predicate this interference. And ultimately to provide a new framework for interference-aware scheduling that would help in improving the performance of HPC applications. This work is the subject for the summer internship for Or\c{c}un Yildiz at ANL (2015).

\paragraph{References}~\\

% print list of publications not from this project but of other relevance
\printbibliography[heading=none,notkeyword=own]

\end{refsection}

\subsubsection{Object-based storage system for HPC systems}
\label{sec.report.RR} % add label, replace xx with your initials

% start new section for referencing, please use the name of your bibfile here instead of report2015 (.bib)
\begin{refsection}[collab_report_RR]

\paragraph{Participants}~\\

Lokman Rahmani, ENS Rennes, PhD student \\
John Jenkins, ANL, Postdoctoral Researcher \\ 
Shane Snyder, ANL, Predoctoral Researcher \\
Rob Ross, ANL, Senior Researcher \\ 
Gabriel Antoniu, Inria, Senior Researcher \\

\paragraph{Research topic and goals}~\\

As we push towards the exascale computing regime, the foundations of
scientific data storage must be revisited. The current model of access, the
POSIX file model, creates barriers to performance by hiding the organization
of data on storage and limiting the ability of higher level I/O software to
express the relationships in scientific data. In order to improve the
efficiency of the storage system, including its use in analysis, applications
and tools must have the means to \emph{capture semantics of data}, such as the
types and dimensions of variables being stored, and to \emph{exploit locality
of access}.
%
We propose to bypass the traditional file model and directly map scientific
datasets onto file system objects, hiding this change in I/O libraries such as
HDF5 and Parallel netCDF, storing semantic information alongside the data, and
exposing data layout such that computation can be co-located with data.
%
In the long term, this additional knowledge and capabilities will facilitate
self-optimization in the storage system.


\paragraph{Results for current year}~\\

\subparagraph{Group Membership:}
Fault response strategies are crucial to maintaining performance and
availability in HPC storage systems, and the first responsibility of a
successful fault response strategy is to detect failures and maintain
an accurate view of group membership. This is a nontrivial problem
given the unreliable nature of communication networks and other system
components.  As with many engineering problems, trade-offs must be
made to account for the competing goals of fault detection efficiency
and accuracy.  Today's production HPC services typically rely on
distributed consensus algorithms and heartbeat monitoring for group
membership.

In this work, we investigate epidemic protocols to determine whether
they would be a viable alternative. Epidemic protocols have been
proposed in previous work for use in peer-to-peer systems, but they
have the potential to increase scalability and decrease fault response
time for HPC systems as well. We focus our analysis on the Scalable
Weakly-consistent Infection-style Process Group Membership (SWIM)
protocol.

We begin by exploring how the semantics of this protocol differ from
those of typical HPC group membership protocols, and we discuss how
storage systems might need to adapt as a result.  We use existing
analytical models to choose appropriate SWIM parameters for an HPC use
case. We then develop a new, high-resolution parallel discrete event
simulation of the protocol to confirm existing analytical models and
explore protocol behavior that cannot be readily observed with
analytical models.  Our preliminary results indicate that the SWIM
protocol is a promising alternative for group membership in HPC
storage systems, offering rapid convergence, tolerance to transient
network failures, and minimal network load~\cite{snyder:swim}.

\subparagraph{Organizing the HPC Storage Community:}
Storage systems are a foundational component of computational,
experimental, and observational science today. The success of
Department of Energy (DOE) activities in these areas is inextricably
tied to the usability, performance, and reliability of storage and
input/output (I/O) technologies. In December 2014, a diverse group of
domain and computer scientists from the Office of Science, the
National Nuclear Security Administration, industry, and academia
assembled in Rockville, Maryland, to review the storage system and
input/output (SSIO) requirements for simulation-driven activities
associated with DOE science, energy, and national security missions
and to assess the state of the art in key storage system and I/O
areas. The activity was organized into three workshops.

The first workshop consisted of a series of talks from six DOE and
NNSA application representatives, many of whom are engaged in exascale
application co-design activities. The talks detailed the
characteristics of the simulation and analysis tasks that the
researchers expect to perform and the I/O characteristics of these
tasks; the anticipated use of new storage technologies such as
nonvolatile memory in accomplishing their science goals; the ways in
which data is organized and searched and how the history of that data
maintained; and the expected impact of issues such as increasing error
rates and technologies such as compression. Following each talk the
group discussed the key points raised and potential areas for future
research and development.

The computing landscape is changing rapidly, and so the second
workshop focused on how other computer science technologies (i.e.,
computer architecture, operating systems and runtimes, networking
systems, workflow systems, data analysis and visualization algorithms,
resilience techniques, and collaboration tools) will influence, and
will be influenced by, future SSIO solutions. Seven talks by computer
science experts in these fields identified the manner in which these
systems are interrelated, and discussion with these experts further
focused attention on key issues.

The information from these two workshops fed into the third workshop,
during which SSIO community members engaged in open discussion on
potential research directions in SSIO to support extreme-scale
simulation-based DOE science. During five interactive sessions, the
participants openly discussed the state of the art in specific SSIO
technology areas and identified challenges and areas where additional
research was needed in these areas~\cite{ross:ssio}.


\paragraph{Visits and meetings}~\\

\begin{itemize}
\item June 2 - June 6: Rob Ross visited KerData in Rennes.
\item June 9 - June 11: 11th workshop of the JLESC held in Nice, France.
\item November 24 - November 26: Meetings for updates and planning were held during the 2nd JLESC 
\item June 29 - July 3: 12th workshop of the JLESC to be held in Barcelona, Spain.
\end{itemize}

\paragraph{Impact and publications}~\\

% print list of publications containing the ``own'' keyword (for publications done within this project and year)
\printbibliography[heading=none,keyword=own]

\paragraph{Person-Month efforts}~\\

\begin{itemize}
\item Rob Ross, ANL 1MM
\item Gabriel Antoniu, INRIA 1MM
\item Lokman Rahmani, INRIA 4MM
\item Shane Snyder, ANL 5MM
\item John Jenkins, ANL 3MM
\end{itemize}

\paragraph{Future plans}~\\

\subparagraph{Exascale System Services:} We define a system service as a long-running, distributed system designed to provide a given capability to multiple users and applications on the platform. Few such services run in existing HPC environments: parallel file systems are one example.

We propose to research and develop elasticity capabilities for
Mercury. This would include a concept of the members of the service
and an ability to add and remove members at runtime. One research
question is whether this should be integrated into Mercury or provided
as an optional component of some sort.

We further propose to build on the existing Mercury components to
develop a prototype service (e.g., a multi-user, resilient
keyword/value store). We will query other LDRD groups to assess what
model is most appropriate and will adjust the details of the service
as appropriate. Testing and evaluation would be performed on the Theta
system.

\paragraph{References}~\\

% print list of publications not from this project but of other relevance
\printbibliography[heading=none,notkeyword=own]

\end{refsection}


\subsection{Parallel programming}

\subsubsection{Energy Efficiency and Load Balancing}
\label{sec.report.jfm} % add label, replace xx with your initials

% start new section for referencing, please use the name of your bibfile here instead of report2015 (.bib)
\begin{refsection}[loadbalancing]

\paragraph{Participants}~\\

\begin{itemize}
\item Rafael Keller Tesser, PhD student UFRGS, associated INRIA team ExaSE
\item Edson Luiz Padoin, PhD student, UFRGS, associated INRIA team ExaSE
\item Philippe Navaux, Professor, UFRGS, associated INRIA team ExaSE
\item Celso Mendes, NCSA
\item Sanjay Kale, UIUC
\item Jean-François Méhaut, Professor, INRIA Grenoble, Corse INRIA team, associated INRIA team ExaSE
\end{itemize}

\paragraph{Research topic and goals}~\\

The power consumption of High Performance Computing (HPC) systems is
an increasing concern as large-scale systems grow in size and,
consequently, consume more energy.  In response to this challenge, we
propose new energy-aware load balancers that aim at reducing the
energy consumption of parallel platforms running imbalanced scientific
applications without degrading their performance. Our research
explores dynamic load balancing, low power manycore platforms and DVFS 
techniques  in order to reduce power consumption.

\paragraph{Results for current year}~\\

{\bf Load balancing}\\

In this work we propose the improvement of the performance and scalability 
of parallel seismic wave models through dynamic load balancing. These models 
suffer from load imbalance for two reasons. First, they add a specific 
numerical condition at the borders of the domain, in order to absorb the 
outgoing energy. The decomposition of the domain into a grid of subdomains, 
which are distributed among tasks, creates load differences between the
tasks that simulate the borders and those responsible for the central 
subdomains. Second, the propagation of waves in the simulated area changes 
the workload on the subdomains on different time-steps. Therefore causing 
dynamic load imbalance. In order to evaluate the use of dynamic load balancing,
we ported a seismic wave simulator to Adaptive MPI, to benefit from its load 
balancing framework. Our experimental results show that dynamic load balancers 
can adapt to load variations during the application’s execution and improve 
performance by 36\%.

This work was presented in the PDP 2014 conference\cite{PDP2014}. An
extended version will be published in the International Journal of
High Performance Computing and Applications (\cite{IJHPCALB}). Laercio
Pilla described most of the load balancers in his
PhD\cite{PhDLaercio}.

\vspace{0.5cm}
{\bf Power consumption}\\

Power consumption is one of the main challenges to achieve Exascale
performance. Current research trends aim at overcoming power
consumption constraints using low-power processors. Although new
processors feature sensors that enable precise power measurements,
they provide different interfaces to collect data, making it difficult
to correlate performance with energy consumption. To overcome this
issue, we developed a platform-independent tool that collects power
and energy data from homogeneous and heterogeneous systems. Using this
tool, it provides a detailed comparison between a low-power processor
(ARM big.LITTLE) and a high performance processor (Intel Sandy
Bridge-EP) using all applications from the NAS parallel benchmarks and
a real-world soil irrigation simulator. The results show that the
average power demand of Intel Sandy Bridge-EP is within $12.6X$ to
$152.4X$ higher than ARM big.LITTLE, whereas its average energy
consumption is within $1.6X$ to $7.1x$ superior. Overall, ARM
big.LITTLE presented a better performance/energy trade-off when it
takes less than $9.2X$ the execution time of Intel Sandy Bridge-EP to
solve the same problem.

This work was published in \cite{IETCDT} and \cite{JPDC}. 

Large-scale simulation of seismic wave propagation
is an active research topic. Its high demand for processing power
makes it a good match for High Performance Computing (HPC).
Although we have observed a steady increase on the processing
capabilities of HPC platforms, their energy efficiency is still
lacking behind. In this work, we analyze the use of a low-power
manycore processor, the MPPA-256, for seismic wave propagation
simulations. First we look at its peculiar characteristics such as
limited amount of on-chip memory and describe the intricate
solution we brought forth to deal with this processor’s idiosyn-
crasies. Next, we compare the performance and energy efficiency
of seismic wave propagation on MPPA-256 to other common-
place platforms such as general-purpose processors and a GPU.
Finally, we wrap up with the conclusion that, even if MPPA-256
presents an increased software development complexity, it can
indeed be used as an energy efficient alternative to current HPC
platforms, resulting in up to 71\% and 81\% less energy than a
GPU and a general-purpose processor, respectively.

This work was presented at the SBAC PAD conference in Paris \cite{SBAC}.

\vspace{0.5cm}

{\bf Load Balancing and Power Saving}\\

In this work, we focus on reducing the energy consumption of
imbalanced applications through a combination of load balancing and
Dynamic Voltage and Frequency Scaling (DVFS). Our strategy employs an
Energy Daemon Tool to gather power information and a load balancing
module that benefits from the load balancing framework available in
the CHARM++ runtime system. We propose two variants of our
energy-aware load balancer (ENERGYLB) to save energy on imbalanced
workloads without considerably impacting the overall system
performance. The first one, called Fine- Grained EnergyLB
(FG-ENERGYLB), is suitable for plat- forms composed of few tens of
cores that allow per-core DVFS. The second one, called Coarse-Grained
EnergyLB (CG-ENERGLB) is suitable for current HPC platforms composed
of several multi-core processors that feature per-chip DVFS.

This work was presented at the HiPC conference \cite{HiPC}.

\paragraph{Visits and meetings}~\\

\begin{itemize}
\item Edson Padoin, November 2014, JLESC Workshop, Chicago
\item Jean-François Méhaut, November Z014, JLESC Workshop, Chicago
\item Brice Videau, June 2015, JLESC Workshop, Barcelona
\item Jean-François Méhaut, June 2015, JLESC Workshop, Barcelona
\end{itemize}


\paragraph{Impact and publications}~\\

% print list of publications containing the ``own'' keyword (for publications done within this project and year)
\printbibliography[heading=none,keyword=own]

\paragraph{Person-Month efforts}~\\

\begin{itemize}
\item Rafael Keller Tesser: 6 PMs
\item Edson Luiz Padoin: 4 PMs
\item Philippe Navaux: 1 PM
\item Celso Mendes:: 0.5 PM
\item Sanjay Kale: 0.25 PM
\item Jean-François Méhaut: 1 PM
\end{itemize}

\paragraph{Future plans}~\\

\begin{itemize}

\item Using simulations (SimGrid, BigSim, Dimemas) for the design and 
analysis of load balancers (Rafael Tesser, Philippe Navaux, Arnaud Legrand, Celso Mendes)

\item Load Balancing and heterogenous platforms/processors (Victor Martinez, 
Fabrice Dupros/BRGM, Philippe Navaux, Jean-François Méhaut)

\end{itemize}


\paragraph{References}~\\

Provide a few bibliographical references here.

% print list of publications not from this project but of other relevance
\printbibliography[heading=none,notkeyword=own]

\end{refsection}

\subsubsection{Enhancing Asynchronous Parallelism in OmpSs with Argobots}
\label{sec.report.pb} % add label, replace xx with your initials

% start new section for referencing, please use the name of your bibfile here
% instead of report2015 (.bib)
\begin{refsection}[report2015_argobot]

\paragraph{Participants}~\\
\begin{itemize}
  \item Pavan Balaji, Argonne National Laboratory, Computer Scientist
  \item Sangmin Seo, Argonne National Laboratory, Postdoctoral Appointee
  \item Huiwei Lu, Argonne National Laboratory, Postdoctoral Appointee
  \item Rosa M. Badia, Barcelona Supercomputing Center, Team leader
  \item Jesus Labarta, Barcelona Supercomputing Center, Computer Science Director
  \item Xavier Teruel, Barcelona Supercomputing Center, Researcher
  \item Vicenc Beltran Querol, Barcelona Supercomputing Center, Senior
Researcher
\end{itemize}

\paragraph{Research topic and goals}~\\

%List research topic and goals.
As future applications on exascale systems are expected to contain billions of
threads or tasks to exploit concurrency provided by the underlying hardware,
parallel programming models need to evolve to efficiently support massive
parallelism with low overhead. OmpSs is a programming model that extends OpenMP
with new directives to support asynchronous parallelism. It enables asynchronous
parallelism by using data-dependencies between different tasks of the
application. Argobots is a low-level infrastructure that supports lightweight
user-level threads (ULTs) and tasks for massive concurrency. It directly
leverages the lowest-level constructs in the hardware and OS, such as
lightweight notification mechanisms, data movement engines, memory mapping, and
data placement strategies.

In this project, we aim at improving asynchronous parallelism support in OmpSs
with Argobots. As a first step, we will evaluate the possible integration of
Argobots into the OmpSs runtime. Since Argobots provides two kinds of work units
(one is a ULT that has context-switching ability, and the other is a tasklet
that is suitable for atomic execution), efficient mapping between OmpSs tasks
and Argobots’ work units will be beneficial to improve performance. If this
evaluation is positive, we will proceed to a prototype implementation of the
integration. Also, we will investigate how OmpSs can take advantage of two-level
parallelism and deterministic threading model in Argobots. Argobots abstracts
its execution model with hierarchical constructs and exposes their execution to
users. OmpSs can exploit this model to optimize multi-level parallelism required
for recursive algorithms. Finally, we will explore scheduling capability of
Argobots in OmpSs. Argobots allows users to write their own scheduler and to
stack different schedulers with different scheduling strategies. We will
implement a locality-aware scheduler in Argobots, which will boost the execution
performance of OmpSs tasks.


\paragraph{Results for current year}~\\

%Write a few paragraphs on main results for current year. Describe relevant
%publications here~\cite{j132,j133}.
We have discussed the feasibility of integrating OmpSs and Argobots. We reviewed
both runtimes and compared their functionalities. The first observation is that
Argobots and OmpSs runtime are overlapped in some basic functionalities and
components, which are strongly embedded in the OmpSs runtime, and thus porting
directly OmpSs runtime on top of Argobots ULTs will add many layers of
components and scheduler overheads. We think that it needs more time and results
to evaluate both runtimes and to come up with appropriate integration
approaches.


\paragraph{Visits and meetings}~\\

%List visits and meetings (planned and done).
We had a telecon meeting on December 15, 2014, where we discussed the current
status of Argobots and OmpSs and interaction between two runtimes. We started
our collaboration after this meeting. We have also exchanged emails to discuss
the integration between Argobots and OmpSs runtime, current issues, and
implementation plans.

We will schedule more telecon meetings for discussion and plan visits as needed.

Planned visits: Not planned yet.


\paragraph{Impact and publications}~\\

% print list of publications containing the ``own'' keyword (for publications
% done within this project and year)
%\printbibliography[heading=none,keyword=own]
This project will have the future impact and contributions as follows:
\begin{itemize*}
  \item Improving the OmpSs programming model with a new lightweight
        threading/tasking library, Argobots.
  \item Exploring lightweight ULTs to support asynchronous parallelism and
        enhancing asynchronous parallelism support in directive-based
        programming models.
  \item Implementing a runtime prototype that integrates Argobots and supports
        OmpSs.
  \item Evaluating performance of the runtime prototype with OmpSs applications.
\end{itemize*}

We plan to publish one or two papers to share our work on interaction between
OmpSs and Argobots with the HPC community and to make our runtime prototype open
source software at the end of the project.


\paragraph{Person-Month efforts}~\\

%This is very important for the JLESC activity report.
%Detail person-months spent by both permanent and temporary researchers
%who worked for the collaboration.
The following are the person-month efforts of the project members spent since
the start of the project.
\begin{itemize}
  \item Pavan Balaji: 0.5
  \item Sangmin Seo: 1
  \item Huiwei Lu: 1
  \item Jesus Labarta: 0.5
  \item Rosa M. Badia: 0.5
  \item Xavier Teruel: 1.5
  \item Vicenç Beltran Querol: 0.5
\end{itemize}


\paragraph{Future plans}~\\

%Describe future plans here.
We will analyze real difficulties behind the interaction between Argobots and
OmpSs runtime and evaluate some integration approaches. Once the evaluation
phase is done, we will design and implement a runtime prototype. After that, we
will optimize the prototype runtime and evaluate it with some OmpSs
applications.


\paragraph{References}~\\

%Provide a few bibliographical references here.
\printbibliography[heading=none,notkeyword=own]
\end{refsection}
\subsubsection{Overlap Communication and Computation with Hybrid MPI+OmpSs}
\label{sec.report.pb2} % add label, replace xx with your initials

\paragraph{Participants}~\\
\begin{itemize}
  \item Pavan Balaji, Argonne National Laboratory, Computer Scientist
  \item Huiwei Lu, Argonne National Laboratory, Postdoctoral Appointee
  \item Sangmin Seo, Argonne National Laboratory, Postdoctoral Appointee
  \item Jesus Labarta, Barcelona Supercomputing Center, Computer Science Director
  \item Rosa M. Badia, Barcelona Supercomputing Center, Team leader
  \item Xavier Teruel, Barcelona Supercomputing Center, Researcher
  \item Vicenc Beltran Querol, Barcelona Supercomputing Center, Senior Researcher
\end{itemize}

\paragraph{Research topic and goals}~\\

%List research topic and goals.

The performance of large-scale scientific applications is often determined by
communication and synchronization. To improve application scalability, we need
asynchronous programming models to overlap communication with computation and
to improve the load balance of applications. OmpSs is a programming model that
extends OpenMP with new directives to support asynchronous parallelism and
heterogeneity. It enables asynchronous parallelism by using data-dependencies
between different tasks of the application. MPI is a de facto standard for
communication among processes on distributed memory systems. Combining OmpSs
and MPI will have the potential to extend asynchronous data-flow execution to
distributed memory systems and improve intra-node data movement compared to
MPI-only model.

The project will be structured in several steps. First, we will evaluate the
possibility to integrate OmpSs into the progress engine of MPI. Currently MPI
supports the interaction with threads. Multiple threads will share the same
progress engine to deal with pending MPI requests.  When a thread makes a
blocking MPI call, it will yield to other threads in order not to block the
progress of the MPI process. When integrating OmpSs, we will need to manage
OmpSs tasks to correctly yield the execution to one another inside the progress
engine. In case this evaluation is positive, the integration will be performed.
Also, in this case, in a second step, we will port applications to use this
hybrid runtime and use existing ones already available at BSC. Hybrid MPI+OmpSs
will give us a new opportunity to explore asynchronous data-flow execution in
distributed algorithms. Synchronization will be minimized  thanks to the task
dependency graph and by enabling asynchronous communications to hide
communication cost, and obtaining better scalability in hybrid applications.
And finally we will evaluate and optimize the runtime and write papers. We plan
to share our findings of MPI+OmpSs with the HPC community with paper and open
source software at the end of the project.

\paragraph{Results for current year}~\\

%Write a few paragraphs on main results for current year. Describe relevant
%publications here~\cite{j132,j133}.
We have started the collaboration since December, 2014. All participants in the
project have met each other on a telecon meeting.

We have discussed the feasibility of integrateing MPI and OmpSs. In previous
work, the hybrid MPI/SMPSs runtime from BSC has already investigated the
benefit of using task-based programming model to overlap communication and
computation, where MPI calls were encapsulated in SMPSs tasks, enabling the
runtime scheduler to reorder the execution of communication tasks in relation
to the computational tasks. In this proposal we plan to evaluate how to extend
the current MPI+OmpSs model  with a deeper integration of OmpSs inside the MPI
progress engine to better support communication overlap. The idea is to have
multiple OmpSs tasks making MPI calls, and the MPI will schedule them
internally to overlap different tasks.

We have started implementing an MPI+OmpSs prototype.

\paragraph{Visits and meetings}~\\

%List visits and meetings (planned and done).
December 15, 2014, telecon meeting: All participants joined this telecon
meeting.  This meeting discussed the current status of MPI and OmpSs, and the
feasibility of the integration.  We started the collaboration of ANL and BSC
after this meeting.

Besides, we have regular Email exchanges discussing the runtime design, current
issues and implemention plans.

We will schedule more telecon meetings for discussion and plan visits as needed.

Planned visits: Not planned yet.

\paragraph{Impact and publications}~\\

% print list of publications containing the ``own'' keyword (for publications done within this project and year)
%\printbibliography[heading=none,keyword=own]

We plan to share our findings of MPI+OmpSs with the HPC community with paper
and open source software at the end of the project.

\paragraph{Person-Month efforts}~\\

% This is very important for the JLESC activity report. 
% Detail person-months spent by both permanent and temporary researchers 
% who worked for the collaboration.

The following are the person-month efforts of the project members spent since
the start of the project.
\begin{itemize}
  \item Pavan Balaji: 0.5
  \item Huiwei Lu: 1
  \item Sangmin Seo: 1
  \item Jesus Labarta: 0.5
  \item Rosa M. Badia: 0.5
  \item Xavier Teruel: 1.5
  \item Vicenc Beltran Querol: 0.5
\end{itemize}

\paragraph{Future plans}~\\

% Describe future plans here.
The first step will be implementing the prototype. We will work on improving
the performance of the prototype and investigate how to overlap the
communication and computation with MPI+OmpSs. The next step would be designing
microbenchmarks to explore the benefits of this hybrid runtime. Also, We plan
to porting HPC application to use MPI+OmpSs for communication/computation
overlapping.

\paragraph{References}~\\
\citation{ICS10-Marjanovic}

% Provide a few bibliographical references here.


\subsection{Automatic differentiation}


\subsubsection{Shared Infrastructure for Source Transformation Automatic Differentiation \ongoing}
\label{sec.report.autom-diff} 

% start new section for referencing, please use the name of your bibfile here instead of report2015 (.bib)
\begin{refsection}[automatic-differentiation]

\paragraph{Participants}

\begin{itemize}
	\item Laurent Hascoet, INRIA, Senior Researcher
	\item Paul Hovland, Argonne National Laboratory, Computer Scientist
	\item Sri Hari Krishna Narayanan, Argonne National Laboratory, Assistant Computer Scientist
\end{itemize}


\paragraph{Research topic and goals}~\\

%List research topic and goals.
Computing derivatives of numerical models is an important task for uncertainty quantification, sensitivity analysis, and numerical optimization. Automatic, or algorithmic,  differentiation (AD) computes derivatives at machine precision by reinterpreting the program that implements a given model. Particularly advantageous is the computation of gradients using the reverse (or adjoint) mode of AD because its computational complexity is a small multiple of the cost of the original program for any size of gradient. AD by source transformation of the model’s program yields the most efficient adjoint computations.
We will investigate challenges for computing derivatives of applications aiming at exascale performance. Data dependencies induced by parallel communications have an impact on the generated adjoint computation and will be a topic of our research. When a numerical model is implemented with asynchronous task parallelism, using graph-based schedulers, we can exploit this high-level view to rearrange the adjoint computation for increased efficiency. Intensive use of dynamic memory management is an important aspect of this topic that we will investigate. 
Given limited resources, it is important for us to avoid duplication of effort. Therefore we will work to establish bridges between the AD tools developed by both teams, namely, OpenAD and Tapenade, to exploit their complementary strengths. The above research on adjoint parallel communications and dynamic memory management will be done in this context. 

\paragraph{Results for current year}~\\

The collaboration is starting.

Timeline :\\
M+6 : A library to handle the adjoint of dynamic memory management primitives.\\
M+12 : An AD platform with two-way connection between OpenAD and Tapenade 

Computer resource needs:\\
   - No big resource needs. Possibly occasional access to a small cluster.

Expected results:\\
Joint publication on adjoint of dynamic memory management\\
Report on the shared infrastructure for OpenAD and Tapenade


%We are currently finish
%Write a few paragraphs on main results for current year. Describe relevant publications here~\cite{aupy:hal-01147155}.

\paragraph{Visits and meetings}~\\
%List visits and meetings (planned and done).
%Visits done:

\paragraph{Impact and publications}~\\

% print list of publications containing the ``own'' keyword (for publications done within this project and year)
\printbibliography[heading=none,keyword=own]


%\begin{itemize*}
%    \item done this
%    \item impact there
%\end{itemize*}

\paragraph{Person-Month efforts}~\\

%This is very important for the JLESC activity report. 
%Detail person-months spent by both permanent and temporary researchers 
%who worked for the collaboration.
\begin{itemize}
	\item {\bf Laurent Hascouet}, permanent researcher, X months
	\item {\bf ...}, ...
\end{itemize}


\paragraph{Future plans}~\\
%Describe future plans here.


\paragraph{References}~\\

%Provide a few bibliographical references here.

% print list of publications not from this project but of other relevance
\printbibliography[heading=none,notkeyword=own]
%\bibfile{mybeautifulproject}
\end{refsection}



\subsubsection{Checkpointing strategies for adjoint computations \ongoing}
\label{sec.report.adjoint} 

% start new section for referencing, please use the name of your bibfile here instead of report2015 (.bib)
\begin{refsection}[adjoint-computations]

\paragraph{Participants}~\\

\begin{itemize}
	\item Guillaume Aupy (INRIA)
	\item Julien Herrmann (INRIA)
	\item Paul Hovland (ANL)
	\item Yves Robert (INRIA)
\end{itemize}


\paragraph{Research topic and goals}~\\
%List research topic and goals.
The need to efficiently compute the derivatives of a function arises frequently
in many areas of scientific computing, including mathematical optimization,
uncertainty quantification, and nonlinear systems of equations. When the first
derivatives of a scalar-valued function are desired, so-called adjoint
computations can compute the gradient at a cost equal to a small constant times
the cost of the function itself, without regard to the number of independent
variables.
In this work we want to optimize the computational time of such functions when
the memory is not sufficient to hold all  necessary data~\cite{Heimbach20051356}.

Previously, optimal algorithms for adjoint computations were known only for a
single level of checkpoints with no writing and reading costs; a well-known
example is the binomial checkpointing algorithm of Griewank and
Walther~\cite{griewank1992achieving,griewank2000algorithm}. Stumm and Walther
extended that binomial checkpointing algorithm to the case of two levels of
checkpoints, but they did not provide any optimality
results~\cite{stumm2009multistage}.

Goals:
\begin{enumerate}
	\item Computing efficient schedules for the offline adjoint problem with two
storage levels (memory and disks);
	\item Computing efficient schedules for the online adjoint problem (when the
number of operations to perform is not known before execution);
	\item Computing efficient schedules in the presence of failures;
	\item Implementing thee algorithms into adjoint computation software.
\end{enumerate}


\paragraph{Results for current year}~\\

We have designed the first optimal algorithm for the offline adjoint problem with two
storage levels. A prototype evaluation showed that we save more than 60\% on
the execution time on realistic adjoint computations~\cite{aupy:hal-01147155}.

%We are currently finish
%Write a few paragraphs on main results for current year. Describe relevant publications here~\cite{aupy:hal-01147155}.

\paragraph{Visits and meetings}~\\
%List visits and meetings (planned and done).
%Visits done:
\emph{Past visits and meetings}
\begin{itemize}
	\item[] Guillaume Aupy visited Paul Hovland at Argonne National Laboratory
from Sept. 2014 to Dec. 2014 .
\end{itemize}
\emph{Planned visits and meetings}
\begin{itemize}
	\item[] Paul Hovland will visit Guillaume Aupy and Yves Robert from
June 22nd, 2015 to June 26th, 2015.
	\item[] Guillaume Aupy and Paul Hovland plan to attend the $17^{th}$
AutoDiff Workshop in July 2015 to present the first results.
\end{itemize}

\paragraph{Impact and publications}~\\

% print list of publications containing the ``own'' keyword (for publications done within this project and year)
\printbibliography[heading=none,keyword=own]


%\begin{itemize*}
%    \item done this
%    \item impact there
%\end{itemize*}

\paragraph{Person-Month efforts}~\\

%This is very important for the JLESC activity report. 
%Detail person-months spent by both permanent and temporary researchers 
%who worked for the collaboration.
\begin{itemize}
	\item {\bf Guillaume Aupy}, temporary researcher, 8 months
	\item {\bf Julien Herrmann}, temporary researcher, 4 months
	\item {\bf Paul Hovland}, permanent researcher, 1 month
	\item {\bf Yves Robert}, permanent researcher, 1 month
\end{itemize}


\paragraph{Future plans}~\\
%Describe future plans here.

As stated above, we aim at achieving the following goals:
\begin{enumerate}
	\item Computing efficient schedules for the online adjoint problem (when the
number of operations to perform is not known before-hand);
	\item Computing efficient schedules in the presence of failures;
	\item Implementing those algorithms into adjoint computation software.
\end{enumerate}

\paragraph{References}~\\

%Provide a few bibliographical references here.

% print list of publications not from this project but of other relevance
\printbibliography[heading=none,notkeyword=own]
%\bibfile{mybeautifulproject}
\end{refsection}


\subsection{To be classified}

\subsubsection{Developer tools for porting \& tuning parallel applications on extreme-scale parallel systems}
\label{sec.report.bjnw} % add label, replace xx with your initials

% start new section for referencing, please use the name of your bibfile here instead of report2015 (.bib)
\begin{refsection}[report2015_bjnw]

\paragraph{Participants}%~\\

\begin{itemize}
\item Brian J.\,N. Wylie, JSC
\item Miwako Tsuji, RIKEN
\item TBD
\end{itemize}


\paragraph{Research topic and goals}~\\

Application developers targeting extreme-scale HPC systems such as
\textsl{JUQUEEN} (BG/Q) and \textsl{Kei} (K~computer) need effective
tools to assist with porting and tuning for these unusual systems (e.g.,
where measurement configuration and analysis result files must be
explicitly staged in and out of batch job partitions). The
\textit{XcalableMP} compilation system (and directive-based
language)~\cite{10.1109/ICPPW.2010.62,10.1109/ICPP.2013.58}  and
\textit{Scalasca}/\textit{Score-P} execution measurement and analysis
tools~\cite{geimer_ea:2010:scalascaarchitecture,knuepfer:2011:scorep}
(using \textit{SIONlib} scalable file
I/O~\cite{frings_ea:2009:parallelio}) are two notable examples of tools
developed by RIKEN and JSC for this purpose. This project proposes to
extend their support for JLESC HPC systems and exploit their
capabilities in an integrated work flow. 

Existing training material will be adapted to collaborators' large-scale
HPC systems, augmented with newly prepared material, and refined for
better uptake based on participant evaluations and feedback. Travel and
accommodation expenses of training presenters to participate in joint
training events (such as VI-HPS Tuning Workshops~\cite{VI-HPS-TWS}) will
be supported. Collaborative work with application developers will assess
the effectiveness of the current (and revised) tools, and help direct
development of new tool capabilities.


\paragraph{Results for current year}~\\

%Write a few paragraphs on main results for current year. Describe relevant publications here~\cite{j132,j133}.

JLESC project proposed (and accepted) in April 2015, and planning
commenced for initial presentation at 3rd JLESC Workshop (Barcelona) in
June 2015.  RIKEN AICS applied to join VI-HPS with a view
to contributing to VI-HPS Tuning Workshops~\cite{VI-HPS-TWS} and Tools Guide~\cite{VI-HPS_ToolsGuide}.
\textit{XcalableMP} training material being translated into English, and
\textit{Scalasca/Score-P} training material being translated into Japanese.


\paragraph{Visits and meetings}~\\

% List visits and meetings (planned and done).

Initial face-to-face meeting planned as part of 3rd JLESC Workshop (Barcelona) and
follow-up at ISC-HPC (Frankfurt am Main) in July.

\paragraph{Visits planned for the next 12 months}~\\

% Planned visits for the next 12 months 
% (number of faculties, duration for each, number of students, duration for each):

Visit of RIKEN AICS researcher to JSC perhaps after ISC-HPC or in December 2015 (1--2 weeks).
Visit of JSC researcher to RIKEN considered for 2016 (perhaps 4 weeks).

\paragraph{Impact and publications}~\\

% print list of publications containing the ``own'' keyword (for publications done within this project and year)
\printbibliography[heading=none,keyword=own]

Further impact
\begin{itemize*}
    \item TBD
\end{itemize*}

\paragraph{Person-Month efforts}~\\

%This is very important for the JLESC activity report. 
%Detail person-months spent by both permanent and temporary researchers 
%who worked for the collaboration.

JSC: 0.0, RIKEN: 0.0.

\paragraph{Future plans}~\\

%Describe future plans here.

The existing integration of \textit{XscalableMP} and \textit{Scalasca}
will be updated to the latest community-developed \textit{Score-P}
instrumentation and measurement infrastructure, and made available for
use on \textit{JUQUEEN} BG/Q and \textit{Kei} K computer.  Example batch
scripts (e.g., to stage-in/out configuration and analysis files) and
associated user documentation will be provided.

\textit{XcalableMP} to be included in VI-HPS Tuning Workshop as part of
PRACE Advanced Training Centre (PATC) curriculum to be held in Germany in spring 2016.
VI-HPS Tuning Workshop to be hosted by RIKEN AICS in Japan
tentatively scheduled for February 2016.

\paragraph{References}~\\

%A few bibliographical references
%\cite{VI-HPS_ToolsGuide,geimer_ea:2010:scalascaarchitecture,knuepfer:2011:scorep,frings_ea:2009:parallelio,10.1109/ICPPW.2010.62,10.1109/ICPP.2013.58}.

% print list of publications not from this project but of other relevance
\printbibliography[heading=none,notkeyword=own]

\end{refsection}


\subsubsection{Scalability Enhancements	to FMM for MD Simulations}
\label{sec.report.ik}

% start new section for referencing, please use the name of your bibfile here instead of report2015 (.bib)
\begin{refsection}[refs_ik]

\paragraph{Participants}~\\
\begin{itemize}
  \item Pavan Balaji (ANL)
  \item Ivo Kabadshow (JSC)
  \item David Haensel (JSC)
\end{itemize}

\paragraph{Research topic and goals}~\\

The goal of this joint-lab cooperation covers the topic of parallel programming. We are especially interested in increasing the scalability (strong scaling) of the Fast Multipole Method (FMM) for very large numbers of ranks. 
FMSolvr is a high-performance FMM library being developed by JSC. However, the current intrinsic parallel scaling limitations stem from process synchronization on large-scale systems. We will investigate weak and delayed synchronization models, tasking approaches and other techniques with MPI-3 and upcoming MPI-4 extensions to alleviate some of these performance bottlenecks.

\paragraph{Results for current year}~\\

The project was initiated at the JLESC meeting in November 2014. 
To provide a consistent interface for measuring and tuning parallel code performance some profound changes had to be made to the code. We started implementing an abstract parallelization layer for the FMSolvr library. This includes a threading approach for intra-node communication as well as a parallelization approach for inter-node communication via MPI. The adopted abstraction layer allows easier replacement/improvement of different synchronization strategies within the code.

Since this is a starting cooperation (\starting) no project publications are available at the moment.
\paragraph{Visits and meetings}~\\

Since this is a starting cooperation no visits have been initiated. The upcoming JLESC meeting at BSC will be used as a first project meeting.

\paragraph{Impact and publications}~\\

none yet.

\paragraph{Person-Month efforts}~\\

20\% of a PhD position (David Haensel, JSC) as well as 5\% of a staff position (Ivo Kabadshow, JSC) have been spent on this activity since December 2014. The efforts are likely to increase, once the code base includes a near-complete parallelization layer.

\paragraph{Future plans}~\\

Next, we want to setup an automatic testing framework for the FMM. This allows us to investigate the scaling bottlenecks of the method without the need to provide a real-world MD dataset. The framework will generate the corresponding input data as well as the required FMM parameter on the fly. After this step is finished an extensive scaling analysis on different HPC platforms will be performed. The results of this analysis are the foundation of all subsequent tuning efforts.


\paragraph{References}~\\

none yet.

% print list of publications not from this project but of other relevance
\printbibliography[heading=none,notkeyword=own]

\end{refsection}


\subsection{Financial report}
\subsubsection{Expense report}
%\input{expense_report}
\subsubsection{Personnel involvement}
%\input{personnel}

\end{document}
