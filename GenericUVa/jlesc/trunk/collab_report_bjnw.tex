
\subsubsection{Developer tools for porting \& tuning parallel applications on extreme-scale parallel systems}
\label{sec.report.bjnw} % add label, replace xx with your initials

% start new section for referencing, please use the name of your bibfile here instead of report2015 (.bib)
\begin{refsection}[report2015_bjnw]

\paragraph{Participants}%~\\

\begin{itemize}
\item Brian J.\,N. Wylie, JSC
\item Miwako Tsuji, RIKEN
\item TBD
\end{itemize}


\paragraph{Research topic and goals}~\\

Application developers targeting extreme-scale HPC systems such as
\textsl{JUQUEEN} (BG/Q) and \textsl{Kei} (K~computer) need effective
tools to assist with porting and tuning for these unusual systems (e.g.,
where measurement configuration and analysis result files must be
explicitly staged in and out of batch job partitions). The
\textit{XcalableMP} compilation system (and directive-based
language)~\cite{10.1109/ICPPW.2010.62,10.1109/ICPP.2013.58}  and
\textit{Scalasca}/\textit{Score-P} execution measurement and analysis
tools~\cite{geimer_ea:2010:scalascaarchitecture,knuepfer:2011:scorep}
(using \textit{SIONlib} scalable file
I/O~\cite{frings_ea:2009:parallelio}) are two notable examples of tools
developed by RIKEN and JSC for this purpose. This project proposes to
extend their support for JLESC HPC systems and exploit their
capabilities in an integrated work flow. 

Existing training material will be adapted to collaborators' large-scale
HPC systems, augmented with newly prepared material, and refined for
better uptake based on participant evaluations and feedback. Travel and
accommodation expenses of training presenters to participate in joint
training events (such as VI-HPS Tuning Workshops~\cite{VI-HPS-TWS}) will
be supported. Collaborative work with application developers will assess
the effectiveness of the current (and revised) tools, and help direct
development of new tool capabilities.


\paragraph{Results for current year}~\\

%Write a few paragraphs on main results for current year. Describe relevant publications here~\cite{j132,j133}.

JLESC project proposed (and accepted) in April 2015, and planning
commenced for initial presentation at 3rd JLESC Workshop (Barcelona) in
June 2015.  RIKEN AICS applied to join VI-HPS with a view
to contributing to VI-HPS Tuning Workshops~\cite{VI-HPS-TWS} and Tools Guide~\cite{VI-HPS_ToolsGuide}.
\textit{XcalableMP} training material being translated into English, and
\textit{Scalasca/Score-P} training material being translated into Japanese.


\paragraph{Visits and meetings}~\\

% List visits and meetings (planned and done).

Initial face-to-face meeting planned as part of 3rd JLESC Workshop (Barcelona) and
follow-up at ISC-HPC (Frankfurt am Main) in July.

\paragraph{Visits planned for the next 12 months}~\\

% Planned visits for the next 12 months 
% (number of faculties, duration for each, number of students, duration for each):

Visit of RIKEN AICS researcher to JSC perhaps after ISC-HPC or in December 2015 (1--2 weeks).
Visit of JSC researcher to RIKEN considered for 2016 (perhaps 4 weeks).

\paragraph{Impact and publications}~\\

% print list of publications containing the ``own'' keyword (for publications done within this project and year)
\printbibliography[heading=none,keyword=own]

Further impact
\begin{itemize*}
    \item TBD
\end{itemize*}

\paragraph{Person-Month efforts}~\\

%This is very important for the JLESC activity report. 
%Detail person-months spent by both permanent and temporary researchers 
%who worked for the collaboration.

JSC: 0.0, RIKEN: 0.0.

\paragraph{Future plans}~\\

%Describe future plans here.

The existing integration of \textit{XscalableMP} and \textit{Scalasca}
will be updated to the latest community-developed \textit{Score-P}
instrumentation and measurement infrastructure, and made available for
use on \textit{JUQUEEN} BG/Q and \textit{Kei} K computer.  Example batch
scripts (e.g., to stage-in/out configuration and analysis files) and
associated user documentation will be provided.

\textit{XcalableMP} to be included in VI-HPS Tuning Workshop as part of
PRACE Advanced Training Centre (PATC) curriculum to be held in Germany in spring 2016.
VI-HPS Tuning Workshop to be hosted by RIKEN AICS in Japan
tentatively scheduled for February 2016.

\paragraph{References}~\\

%A few bibliographical references
%\cite{VI-HPS_ToolsGuide,geimer_ea:2010:scalascaarchitecture,knuepfer:2011:scorep,frings_ea:2009:parallelio,10.1109/ICPPW.2010.62,10.1109/ICPP.2013.58}.

% print list of publications not from this project but of other relevance
\printbibliography[heading=none,notkeyword=own]

\end{refsection}
