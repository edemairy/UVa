\subsubsection{Enhancing Asynchronous Parallelism in OmpSs with Argobots}
\label{sec.report.pb} % add label, replace xx with your initials

% start new section for referencing, please use the name of your bibfile here
% instead of report2015 (.bib)
\begin{refsection}[report2015_argobot]

\paragraph{Participants}~\\
\begin{itemize}
  \item Pavan Balaji, Argonne National Laboratory, Computer Scientist
  \item Sangmin Seo, Argonne National Laboratory, Postdoctoral Appointee
  \item Huiwei Lu, Argonne National Laboratory, Postdoctoral Appointee
  \item Rosa M. Badia, Barcelona Supercomputing Center, Team leader
  \item Jesus Labarta, Barcelona Supercomputing Center, Computer Science Director
  \item Xavier Teruel, Barcelona Supercomputing Center, Researcher
  \item Vicenc Beltran Querol, Barcelona Supercomputing Center, Senior
Researcher
\end{itemize}

\paragraph{Research topic and goals}~\\

%List research topic and goals.
As future applications on exascale systems are expected to contain billions of
threads or tasks to exploit concurrency provided by the underlying hardware,
parallel programming models need to evolve to efficiently support massive
parallelism with low overhead. OmpSs is a programming model that extends OpenMP
with new directives to support asynchronous parallelism. It enables asynchronous
parallelism by using data-dependencies between different tasks of the
application. Argobots is a low-level infrastructure that supports lightweight
user-level threads (ULTs) and tasks for massive concurrency. It directly
leverages the lowest-level constructs in the hardware and OS, such as
lightweight notification mechanisms, data movement engines, memory mapping, and
data placement strategies.

In this project, we aim at improving asynchronous parallelism support in OmpSs
with Argobots. As a first step, we will evaluate the possible integration of
Argobots into the OmpSs runtime. Since Argobots provides two kinds of work units
(one is a ULT that has context-switching ability, and the other is a tasklet
that is suitable for atomic execution), efficient mapping between OmpSs tasks
and Argobots’ work units will be beneficial to improve performance. If this
evaluation is positive, we will proceed to a prototype implementation of the
integration. Also, we will investigate how OmpSs can take advantage of two-level
parallelism and deterministic threading model in Argobots. Argobots abstracts
its execution model with hierarchical constructs and exposes their execution to
users. OmpSs can exploit this model to optimize multi-level parallelism required
for recursive algorithms. Finally, we will explore scheduling capability of
Argobots in OmpSs. Argobots allows users to write their own scheduler and to
stack different schedulers with different scheduling strategies. We will
implement a locality-aware scheduler in Argobots, which will boost the execution
performance of OmpSs tasks.


\paragraph{Results for current year}~\\

%Write a few paragraphs on main results for current year. Describe relevant
%publications here~\cite{j132,j133}.
We have discussed the feasibility of integrating OmpSs and Argobots. We reviewed
both runtimes and compared their functionalities. The first observation is that
Argobots and OmpSs runtime are overlapped in some basic functionalities and
components, which are strongly embedded in the OmpSs runtime, and thus porting
directly OmpSs runtime on top of Argobots ULTs will add many layers of
components and scheduler overheads. We think that it needs more time and results
to evaluate both runtimes and to come up with appropriate integration
approaches.


\paragraph{Visits and meetings}~\\

%List visits and meetings (planned and done).
We had a telecon meeting on December 15, 2014, where we discussed the current
status of Argobots and OmpSs and interaction between two runtimes. We started
our collaboration after this meeting. We have also exchanged emails to discuss
the integration between Argobots and OmpSs runtime, current issues, and
implementation plans.

We will schedule more telecon meetings for discussion and plan visits as needed.

Planned visits: Not planned yet.


\paragraph{Impact and publications}~\\

% print list of publications containing the ``own'' keyword (for publications
% done within this project and year)
%\printbibliography[heading=none,keyword=own]
This project will have the future impact and contributions as follows:
\begin{itemize*}
  \item Improving the OmpSs programming model with a new lightweight
        threading/tasking library, Argobots.
  \item Exploring lightweight ULTs to support asynchronous parallelism and
        enhancing asynchronous parallelism support in directive-based
        programming models.
  \item Implementing a runtime prototype that integrates Argobots and supports
        OmpSs.
  \item Evaluating performance of the runtime prototype with OmpSs applications.
\end{itemize*}

We plan to publish one or two papers to share our work on interaction between
OmpSs and Argobots with the HPC community and to make our runtime prototype open
source software at the end of the project.


\paragraph{Person-Month efforts}~\\

%This is very important for the JLESC activity report.
%Detail person-months spent by both permanent and temporary researchers
%who worked for the collaboration.
The following are the person-month efforts of the project members spent since
the start of the project.
\begin{itemize}
  \item Pavan Balaji: 0.5
  \item Sangmin Seo: 1
  \item Huiwei Lu: 1
  \item Jesus Labarta: 0.5
  \item Rosa M. Badia: 0.5
  \item Xavier Teruel: 1.5
  \item Vicenç Beltran Querol: 0.5
\end{itemize}


\paragraph{Future plans}~\\

%Describe future plans here.
We will analyze real difficulties behind the interaction between Argobots and
OmpSs runtime and evaluate some integration approaches. Once the evaluation
phase is done, we will design and implement a runtime prototype. After that, we
will optimize the prototype runtime and evaluate it with some OmpSs
applications.


\paragraph{References}~\\

%Provide a few bibliographical references here.
\printbibliography[heading=none,notkeyword=own]
\end{refsection}