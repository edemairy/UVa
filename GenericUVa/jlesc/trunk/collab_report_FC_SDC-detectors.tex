
\subsubsection{New Techniques to Design Silent Data Corruption Detectors}
\label{Cappello-SDC-detectors} % add label, replace xx with your initials

% start new section for referencing, please use the name of your bibfile here instead of report2015 (.bib)
\begin{refsection}[collab_report_FC_SDC-detectors]

\paragraph{Participants}

\begin{itemize*}
 \item Omer Subasi, BSC, Ph. D. Student 
 \item Leonardo Bautista Gomer, ANL, Postdoc 
 \item Sheng Di, ANL, Postdoc 
 \item Franck Cappello, ANL, Senior Computer Scientist
 \item Jesus Labarta, BSC, Director of Computer Science Department
 \item Osman Unsal, BSC, Senior Researcher 
 \end{itemize*}

\paragraph{Research topic and goals}~\\

Supercomputers allow scientists to study natural phenomena by means of computer simulations. Next-generation machines are expected to have more components and, at the same time, consume several times less energy per operation. These trends are pushing supercomputer construction to the limits of miniaturization and energy-saving strategies. Consequently, the number of soft errors is expected to increase dramatically in the coming years. While mechanisms are in place to correct or at least detect some soft errors, a significant percentage of those errors pass unnoticed by the hardware. Such silent errors are extremely damaging because
they can make applications silently produce wrong results. In this project we propose multiple corruption detectors. A first type of detectors leverages certain properties of high-performance computing applications in order to detect silent errors at the application level. This technique detects corruption solely based on the behavior of the application datasets and is application-agnostic. We propose and we couple them to work together in a fashion transparent to the user. A second type of detectors is based on replication at task level.

\paragraph{Results for current year}~\\

This collaboration is starting. The two teams have a strong record of publications in this domain ~\cite{Leo-PPoPP2014,Edu-HPDC2015,Di-CCGRID2015,Edu-SC2014}.
The collaboration will be implemented by the 1 week visit of Franck Cappello to BSC and the 3 months visit of Omer Subasi at Argonne from mid July 2015. 

\paragraph{Visits and meetings}

\begin{itemize*}
 \item Franck Cappello (ANL) to BSC from July 6th to July 10th, 2015.
 \item Omer Subasi (BSC) to ANL from July 15th to October 15, 2015.
\end{itemize*}


\paragraph{Impact and publications}~\\

% print list of publications containing the ``own'' keyword (for publications done within this project and year)
\printbibliography[heading=none,keyword=own]

Further impact
\begin{itemize*}
    \item none yet
% \item impact there
\end{itemize*}

\paragraph{Person-Month efforts}~\\

Efforts will be spent from July 2015.

\paragraph{Future plans}~\\

After the two visits planned from July 2015, we plan to work on a join-paper, submit if to IPDPS, PPoPP or CCGRID and then present the results during the November JLESC workshop.

\paragraph{References}~\\

Provide a few bibliographical references here.

% print list of publications not from this project but of other relevance
\printbibliography[heading=none,notkeyword=own]

\end{refsection}
